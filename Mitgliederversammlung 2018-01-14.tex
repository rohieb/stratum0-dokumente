\documentclass{s0minutes}
\usepackage[utf8]{inputenc}
\usepackage[ngerman]{babel}
\usepackage{longtable}
\usepackage{booktabs} % professional tables
\usepackage{multicol}
\usepackage{wasysym}  % for \diameter
\usepackage{textcomp} % for €

\meetingminutes{\generalassembly}{14. Januar 2018}{14:00}{Stratum~0, Hamburger
Straße 273 A2, Braunschweig}{32 stimmberechtigte Mitglieder,\\ & keine nicht
stimmberechtigten Mitglieder,\\ & ein Gast}{}{rohieb}

\title{9.\, Mitgliederversammlung}

% Sowas wie "Präsentati-on" braucht doch niemand -.-
% FIXME \hyphenpenalty=10000

\begin{document}
\maketitle

%%%%%%%%%%%%
%% TOP 0  %%
%%%%%%%%%%%%
\section{Protokoll-Overhead}
\begin{description}
\raggedright
  \item[Eröffnung der Versammlung] durch den Vorstandsvorsitzenden um 14:15
  \item[Wahl der Versammlungsleitung:] Kasalehlia, durch Handzeichen, kein Einspruch
  \item[Wahl der Protokollführung:] rohieb, durch Handzeichen, kein Einspruch
  \item[Quoren:] \quad
    \begin{itemize}[leftmargin=0cm]
      \item zum Tag der Mitgliederversammlung hat der Verein insgesamt 102
        Mitglieder, davon 92 ordentliche Mitglieder und 10 Fördermitglieder
      \item anwesend sind 32 stimmberechtigte, ordentliche Mitglieder
      \item 23\% der ordentlichen Mitglieder = \emph{21{,}16 Mitglieder für
        Beschlussfähigkeit}
      \item 50\% der anwesenden, stimmberechtigten Mitglieder = \emph{16 Mitglieder 
        für die Annahme eines Antrags}
    \end{itemize}
  \item[Beschlussfähigkeit:] 32 von geforderten 21{,}16 stimmberechtigten
    Mitgliedern sind anwesend, die Versammlung ist damit beschlussfähig.
  \item[Notation für Abstimmungen:] (Pro-Stimmen/Contra-Stimmen/Enthaltungen)
\end{description}

%%%%%%%%%%%%%
%% TOP 1   %%
%%%%%%%%%%%%%
\section{Berichte}

%%%%%%%%%%%%%
%% TOP 1.1 %%
%%%%%%%%%%%%%
\subsection{Jahres(abschnitts)bericht}

larsan und chrissi\^{} geben nochmal einen Überblick über das gesamte vergangene
Jahr. Die Präsentation dazu (mit sehr vielen Bildern) ist auf der Homepage
verfügbar.%
\footnote{\url{https://stratum0.org/wiki/Datei:JahresberichtMV2018.pdf}}

\paragraph{Gemeinnützigkeit}

Die Gemeinnützigkeit wurde dem Verein vom Finanzamt Braunschweig--Wilhelmstraße
am 19. April 2017 bestätigt, der Verein ist damit von der Körperschafts- und
Gewerbesteuer befreit und darf Zuwendungsbescheinigungen für Spenden und
Mitglieds\-beiträge ausstellen.  Die nächste Prüfung findet 2019 statt.

\paragraph{Gruppen und Veranstaltungen}

Im Space finden regelmäßig Vorträge statt, im vergangenen Jahr 35 Stück, und
seit Gründung insgesamt 215 Vorträge. In den letzten Monaten wurden die
Vorträge in der Regel auch aufgezeichnet. Weitere Verbesserungen des
Aufzeichnungs-Workflows sind geplant. larsan weist darauf hin, dass später am
Abend noch Vorträge stattfinden.

Insgesamt fanden im Space 25 CoderDojos statt. In Zukunft werden diese immer am
vorletzten Samstag im Monat stattfinden. Im November gab es einen Artikel
in der Braunschweiger Zeitung, seitdem sind die Veranstaltungen ausgebucht und
raummäßig am Limit. Mentoren sind immer gerne gesehen, allgemeines Vorwissen ist
dabei kaum vonnöten. Bei Interesse kann auch nochmal eine Vorbereitungsrunde für
Mentoren stattfinden.

Das lokale Freifunk-Projekt wird momentan rebootet.

Neu im Space ist das Projekt Labdoo, in dessen Rahmen Laptops gesammelt und in
infrastrukturschwache Länder verschifft werden.

Die Braunschweiger Linux-User-Group hat sich im Sommer bei uns getroffen, sie
sind jetzt ins Haus der Talente in der Weststadt umgezogen.

Die örtliche Digitalcourage-Gruppe trifft sich jeden zweiten Donnerstag im
Space, und plant demnächst eine Cryptoparty.

Das Malkränzchen, ein offener Kreativabend, findet jeden Montagabend statt.

Der Braunschweiger Kopter Club trifft sich jeden zweiten Dienstag zum
Stammtisch, und besteht aus relativ vielen Leuten. Sie haben auch wieder Geld
gespendet dieses Jahr.

Das Animereferat pausiert momentan, Captain's Log findet regelmäßig statt.

Das Stratum~0 Phone Operation Center (SPOC) hat jetzt eine eigene
DECT-Basestation passend zur gespendeten Alcatel-Telefonanlage. Diese läuft
wegen des hohen Stromverbrauchs nicht permanent, kann aber bei Veranstaltungen
eingesetzt werden.

Das Konzept "`Vegan Academy"' wurde inzwischen auch vom Backspace in Bamberg
erfolgreich eingesetzt.

\paragraph{Öffentlichkeitsarbeit}

Stratum~0 war dieses Jahr im September auf dem "`Markt der Kreativen"' der
KreativRegion Braunschweig vertreten, auf der Maker Faire Hannover gab es
hingegen dieses Jahr mangels Initiative keine Präsenz von uns.

\paragraph{Infrastruktur}

Ein neuer Router (PC-Engines APU2C4, siehe Wiki%
\footnote{\url{https://stratum0.org/wiki/strapu}}) wurde angeschafft, dieser
soll in Zukunft auch mehrere Serverdienste im Space übernehmen. Im Moment läuft
dort auch ein Test-Setup mit UniFi-Access-Points für besseres WLAN im Space.

\paragraph{Marketing}

Wir sind nach wie vor Teil des Gründerquartieres der Stadt Braunschweig. Dazu
gibt es nun auch die \emph{Start-Up-Map Niedersachsen}%
\footnote{\url{https://startup.nds-business-map.de}}, worauf wir verzeichnet
sind.

Freifunk Braunschweig wurde mit 4{.}500\,€ durch die Stadt Braunschweig
(Braunschweig Zukunft GmbH) gefördert. Das Geld soll für ein eigenes
Event-WLAN-Setup mit 16 Accesspoints ausgegeben werden, und es ist der Aufbau
einer Continuous-Integration-Infrastruktur für die Freifunk-Firmware geplant.

\paragraph{Community}

Einige unserer Entitäten waren dieses Jahr auf dem EasterHegg, der GPN, dem SHA,
und dem Hackover. Das Hacken Open Air wurde von uns selbst organisiert, larsan
zeigt Fotos, und beschreibt sie kurz: \emph{Zelt, Zelt, Zelt, Zelt, Teppich,
Dusche, tolle Dusche}. Es waren am Ende insgesamt etwa 130 Leute anwesend, das
Wetter war durchwachsen. Es ist geplant, ein solches Event zu wiederholen,
Details werden demnächst in einem Orgatreffen ausgearbeitet.

Auf dem 34C3 war Stratum~0 wieder mit dem Mensadisplay vertreten, wir haben 150
Ticket-Vouchers generiert. larsan schlägt vor, nächstes Jahr früher mit der
Planung zu beginnen, und betont, dass es dafür nicht erforderlich ist, Teil des
Vorstandes zu sein.
Für den nächsten Congress gibt es die Überlegung, mit anderen unabhängigen
Hackerspaces einen eigenen "`Orbit"' zu gründen.

An Silvester waren 20 Entitäten im Space, es wurde über Brandschutz diskutiert
und Sicherungen getestet.

\paragraph{Diverses} in Kürze:

\begin{itemize}
  \item Das neue LED-Licht fällt nicht mehr in der Stromstatistik auf.
  \item An Spacebauabenden wurden diverse Infrastrukturdinge gehackt.
  \item VDSL100 wird uns weiterhin vom Provider versagt.
  \item Das Shenzhen-Portal wächst.
  \item Stratumnews werden weitergeführt.
  \item Es fanden Parties für Enno und Marudor statt.
  \item Es wurden Bier, Wurst, Sushi, und weiteres Essen und Trinken hergestellt.
  \item Die Spendendose ist verschwunden, es wurde Anzeige erstattet und eine
    neue gebaut (mit Kensington-Schloss).
  \item Selgros- und Metrokarten sind für Mitglieder verfügbar.
\end{itemize}

\paragraph{Zukunft und Ausblick}

Nach verbreiteter Wahrnehmung ist der Space zu klein, und das Problem wächst.

Die Space-Suche in der Umgebung gestaltet sich schwierig. Der Serverraum von
G\ae{}rtner und das Archiv neben uns sind nicht verfügbar, die Räumlichkeiten im
C4-Gebäudeteil des Schimmelhofs wurden von uns besichtigt, aber dann plötzlich
anderweitig vermietet, das IEK unten im Haus ist nach Wasserschaden umgezogen,
deren Räume werden aber vermutlich vom Staatstheater genutzt werden, und kosten
9€ pro Quadratmeter, was momentan nicht diskutabel scheint. Der Wasserschaden
wird von der Versicherung übernommen, und gilt nicht als Mangel für eine
Mietminderung. Laut Kontakt mit der Verwaltung gibt es im Schimmelhof gerade
keine geeigneten Räume.

Die Werkstatt bekam eine Festool-Sachspende, außerdem wurden ein Schweißgerät
von EWM, eine Standbohrmaschine, eine Akku-Stichsäge und ein Akkubohrer
angeschafft.

Die Selbstsicht des Vorstandes wird erläutert: Er sieht sich hauptsächlich als
Verwaltungsorgan, nicht als Regierung, sondern eher als "`erweiterter
Briefkasten"', alles andere sollte aus der Basis kommen.  Die Mitglieder werden
aufgefordert, im Space coole Dinge zu tun, und nicht immer den Vorstand
entscheiden zu lassen.  Es gibt regelmäßige Arbeitstreffen zur Geschäftsführung,
das prinzipiell offen für alle Mitglieder ist. Mögliche Aufgaben dabei sind
z.~B.: Tooling für Finanzauswertung/-automatisierung bauen, Kontakte mit Firmen
und Sponsoren herstellen.

Zum Schluss einige Screenshots von glücklichen Twitter-Benutzern, die es bei uns
toll fanden. Außerdem Danksagungen an unsere diesjährigen Firmenspender:
Triology, Pengutronix, UCware, EWM AG, Festool, in-tech und quuxLogic.


%%%%%%%%%%%%%
%% TOP 1.2 %%
%%%%%%%%%%%%%
\subsection{Finanzbericht}

Emantor gibt als Schatzmeister einen Überblick über die Finanzen im
vergangenen Dreimonatszeitraum seit der letzten Mitgliederversammlung. Der
vollständige Bericht ist als Präsentation auf der Homepage zu finden%
\footnote{\url{https://stratum0.org/wiki/Datei:Finanzbericht_MV_2018.pdf}}
und wird hier auszugsweise mit den mündlichen Anmerkungen wiedergegeben.

\paragraph{Kontenübersicht}

Die Spendenkonten dienen nur noch zur Erfassung der Herkunft der Spenden, ihre
Zweckbindung wurde inzwischen durch die Aufschrift "`nicht zweckgebunden"'
aufgehoben, damit die Spenden flexibel (innerhalb des Vereinszwecks) verteilt
werden können.

Das Girokonto enthält auch die Rückstellungen, die später aufgelistet werden.

\begin{longtable}{ll}
  \textbf{Nr.} & \textbf{Name} \\
  \midrule
  \endfirsthead
  100 & Barkasse \\
	\quad 100-1 & Spenden für 3D-Drucker Filament \\
	\quad 100-2 & Pfand für physische Schlüssel \\
	\quad 100-3 & Spenden für Plotter-Material \\
	\quad 100-4 & Spenden für Stick-Material \\
	\midrule
	101 & Erstattungskasse Verbrauchsmaterial \\
	\midrule
	102 & Matekasse \\
	\midrule
	200024917 & Giro \\
	\quad 200024917-1 & Rückstellungen Girokonto \\
	\midrule
\end{longtable}

\paragraph{Neues seit 2017}
Es wurde sich zwar relativ viel zur Umsetzung vorgenommen (Automatisierung,
Regressionstests der Buchhaltung), durch den relativ kurzen Zeitraum seit der
letzten Mitgliederversammlung ist aber noch nichts davon umgesetzt worden.

Allerdings haben sich die verschwundenen 120€, deren Fehlen bei der
Kassenprüfung 2017 bemängelt worden waren, bei der letzten Kassenrüfung wieder
angefunden.

\paragraph{Überblick über die Finanzen}
Die Zahlen beziehen sich auf den Zeitraum 1. September 2017 bis 9. Januar 2018.

\begin{longtable}{lr>{\textcolor{red}\bgroup}r<{\egroup}}
  \textbf{Bereich} & \textbf{Einnahmen [€]} & \textbf{Ausgaben [€]} \\
  \midrule
  \endfirsthead
  \multicolumn{3}{c}{\emph{(Fortsetzung von vorheriger Seite)}} \\
  \\
  \textbf{Bereich} & \textbf{Einnahmen [€]} & \textbf{Ausgaben [€]} \\
  \midrule
  \endhead
  \\
  \multicolumn{3}{c}{\emph{(Fortsetzung auf nächster Seite)}} \\
  \endfoot
  \endlastfoot
  Ideeller Bereich: Allgemein       &               &                \\
  \quad davon Mitgliedsbeiträge     &               &                \\
  \quad davon Spenden               &               &                \\
  \quad davon allgemeine Ausgaben   &               &                \\
  \quad davon Kontoführungsgebühren &               &                \\
  \quad davon Vereinsserver         &               &                \\
  \quad davon Bekleidung            &               &                \\
  \midrule
  Ideeller Bereich: Projekte        &               &                \\
  \quad davon Bastelmaterial        &               &                \\
  \quad davon Stickmaschine         &               &                \\
  \quad davon Schneidplotter        &               &                \\
  \quad davon 3D-Drucker            &               &                \\
  \quad davon Freifunk              &               &                \\
  \quad davon CoderDojo             &               &                \\
  \midrule
  Ideeller Bereich: Space           &               &                \\
  \quad davon Rundfunkgebühr        &               &                \\
  \quad davon Miete und Nebenkosten &               &                \\
  \quad davon Verbrauchsmaterial    &               &                \\
  \quad davon Einrichtung           &               &                \\
  \midrule
  Vermögensverwaltung               &               &                \\
  \midrule
  Zweckbetriebe                     &               &                \\
  \midrule
  Wirtschaftlicher Geschäftsbetrieb &               &                \\
  \quad (Matekasse) && \\
  \midrule
  Mankobuchungen                    &               &                \\
  \midrule\midrule
  Gesamt:                           &               &                \\

  \textbf{Gewinn/Verlust Gesamt:}   &               &                \\
\end{longtable}

%%%%%%%%%%%%%
%% TOP 1.3 %%
%%%%%%%%%%%%%
\subsection{Bericht der Rechnungsprüfer}

shoragan und Angela berichten von der Kassenprüfung. Die Probleme der letzten
Kassenprüfung im September haben sich alle geklärt. Bei der aktuellen Prüfung
sind kleinere Sachen aufgefallen, z.~B. ist es vorgekommen, das Umlaufbeschlüsse
nur auf dem Papierbeleg statt auch in der Datenbank referenziert wurden.
Insgesamt gibt es aber keine Mängel an der Buchführung, und die Rechnungsprüfer
empfehlen eine Entlastung des Schatzmeisters.

{ \itshape \vspace{2\baselineskip}
  Zwei verspätete Mitglieder werden nachträglich akkreditiert:
  \begin{itemize}[nosep]
    \item anwesend sind nun 34 stimmberechtigte, ordentliche Mitglieder
    \item 50\% der anwesenden, stimmberechtigten Mitglieder = \textbf{17 Mitglieder 
      für die Annahme eines Antrags}
  \end{itemize}
}

%%%%%%%%%%%%%
%% TOP 2   %%
%%%%%%%%%%%%%
\section{Entlastung des Vorstands}

Bei der letzten Mitgliederversammlung wurde der Vorstand nicht entlastet, die
Probleme wurden aber behoben.

Auf die Frage hin, ob jemand möchte, dass die Vorstände einzeln entlastet
werden, meldet sich niemand.

Die Versammlungsleitung bittet um Entlastung des Vortandes für die gesamte
Vergangenheit, für die der Vorstand noch nicht entlastet wurde (also die letzten
13 Monate). Es wird per Handzeichen abgestimmt.

\begin{resolution}{MV 2018-01}{\vote{\adopted}{28}{0}{6}}{Entlastung des
  Vorstandes}{}
  Alle Vorstandsmitglieder enthalten sich.
\end{resolution}

Der Vorstand ist damit für die letzten 13 Monate entlastet.

%%%%%%%%%%%%%
%% TOP 3   %%
%%%%%%%%%%%%%
\section{Wahlen}

%%%%%%%%%%%%%
%% TOP 3.1 %%
%%%%%%%%%%%%%
\subsection{Wahl des Vorstandes}

Als Wahlleitung wird gecko durch Handzeichen ohne Gegenstimmen oder
Enthaltungen gewählt. Die Versammlungsleitung wird an die Wahlleitung übergeben.

Die Wahlleitung erklärt den Wahlmodus:
\begin{itemize}
  \item Es handelt sich um Zustimmungswahl (Approval Voting) mit Stimmzetteln.
    Für jede Entität auf dem Wahlzettel darf also ein Kreuz oder kein Kreuz
    gesetzt werden.
  \item Es gibt ein Quorum von 50\%: Die Person, die das Quorum erreicht und die
    meisten Stimmen hat, ist gewählt.
  \item Bei Beisitzern mit mehreren Posten werden die Posten mit den Personen,
    die das Quorum schaffen, in absteigender Reihenfolge der Stimmen besetzt.
  \item Falls niemand das Quorum schafft, findet eine Nachwahl für die
    notwendigen Posten (Vorstandsvorsitzender, stellv. Vorsitzender und
    Schatzmeister) statt.
  \item Für alle Posten findet ein gemeinsamer Wahlgang statt, dann wird von
    "`oben"' nach "`unten"' besetzt: Wer als Vorstandsvorsitzender schon gewählt
    wurde, fällt für den Posten des stellv. Vorsitzenden weg, ein gewählter
    stellv. Vorsitzender fällt für den Posten des Schatzmeisters weg, ein
    gewählter Schatzmeister fällt für die Posten der Beisitzer weg.
  \item Zur Enthaltung wird der Stimmzettel nicht abgegeben. Ein großes Kreuz
    über den gesamten Block eines einzelnen Vorstandspostens wird von der
    Wahlleitung als Enthaltung für diesen einzelnen Posten gewertet.
\end{itemize}

Die Wahlleitung übergibt zurück an die Versammlungsleitung.

%%%%%%%%%%%%%
%% TOP 3.2 %%
%%%%%%%%%%%%%
\subsection{Wahl der Rechnungsprüfer}

	* aktuell: shoragan, angela, beide würden weitermachen.
	* fordert jemand, dass wir die rechnungsprüfer neu wählen? es müssen laut satzung genau zwei sein. niemand fordert das, rechnungsprüfer bleiben im amt.

%%%%%%%%%%%%%
%% TOP 3.3 %%
%%%%%%%%%%%%%
\subsection{Wahl der Vertrauensperson}

	* daniel stulle würde es weiter machen, ist aber im letzten jahr nichts passiert ...
	* [frage, was die vertrauensperson macht. manche wussten auch nicht, dass sie existiert.]
		* vorstand hat das amt vorgeschlagen als zusätzlichen ansprechpartner für interpersonelle konflikte im space neben dem vorstand.
	* möchte jemand dieses amt neu wählen? nein. dstulle bleibt weiter im amt.
	* anmerkung, dass der posten besser beworben wird.
	* anmerkung, dass ein bericht der vertrauensperson auf der MV sinnvoll ist.

%%%%%%%%%%%%%
%% TOP 4   %%
%%%%%%%%%%%%%
\section{Sonstiges}

%%%%%%%%%%%%%
%% TOP 4.1 %%
%%%%%%%%%%%%%
\subsection{Vergrößerung des Spaces}

		* siehe oben: platz ist mangelware, besonders bei veranstaltungen, aber auch schonmal unter der woche bei parallelen events.
		* viele entitäten haben interesse an der vergrößerung.
		* optionen: den jetzigen space behalten und zusätzliche flächen anmieten, oder aktuellen space aufgeben und mehr platz an einem stück auftun.
			* space hier behalten: vorteil: hier ist es schon eingerichtet. nachteil: aktuell keine möglichkeit zur vergrößerung.
			* neuen space am stück suchen: mehr optionen, muss aber bei null begonnen werden, und es ist doch einiges an arbeit, wieder den aktuellen wohlfühlfaktor zu erhalten.
		* aktuell gute finanzielle ausgangslage für suche nach neuen flächen. aktuell sind wir aber in sehr günstigen räumlichkeiten, ist nicht einfach, was äquivalentes mit entsprechender lage zu finden.
		* frage: können wir uns finanziell mehr leisten? gibt es mitglieder, die finanzielles risiko mittragen würde?
			* äußerung: viel arbeit und viel geld in den aktuellen space gesteckt, man müsste erstmal sehr viel investieren, um einen neuen space hochzuziehen, größenordnung 10.000€.
				* wenn man in zwei jahren wieder zu wenig geld hat, ist es schade, noch mal umziehen zu müssen.
			* äußerung: wir haben hier im aktuellen space sehr viel selbst erarbeitet.
			* schatzmeister: 7000€ seit bezug space2.0 für einrichtung ausgegeben, teilweise aber auch wiederbenutzbar im neuen space.
			* äußerung: ähnliche situation bei umzug in space2.0, nicht genug mitglieder damals für finanzierung. frage damals: wieviel ptenzial zum wachstum? ähnliche frage sollte man heute auch stellen.
			* äußerung: gruppen im space örtlich auseinander ziehen? bspw auch protohaus für maker-zeug. 
			* gegenäußerung: alles an einem ort macht uns schon aus. in den space kommen, was essen, dann an die kappsäge, es ist halt bequem.
			* äußerung: gefühl, es ist kein platz für mich im space, weil überall was rumsteht/von anderen leuten besetzt ist
			* äußerung zu mitgliederwachstum: wachstum im moment mit null werbung erreicht, oder? (naja.) vermutlich mehr (oder überhaupt) werbung -> mehr mitglieder
			* frage zu IEK-räumlichkeiten unten? antwort von reneger: siehe oben. eher nicht verfügbar, und teuer als erweiterungsfläche.
			* als wir space1.0 bezogen haben, war space2.0 auch schon frei. hausverwaltungsverpeilung existiert. reneger telefoniert viel mit der verwaltung, ziel ist einmal im monat :)
				* altes tanzsportzentrum wurde uns angeboten, stimmung war aber eher "nicht schon wieder in den dritten stock"
		* meinungsbild: wollen wir uns vergrößern? viel zustimmung, vermutlich sogar mehrheit.
		* falls es konkretes gibt, wird es wieder eine MV dazu geben.
		* frage: wie sind die prioritäten? Uni-nah? 
			* ja, location ist ein großer faktor. innenstadt-/uninah und nah an der autobahn ist ein großes plus für uns.
		* meinungsbild: wollen wir als verein das risiko eingehen, monatliche verpflichtung aus (nicht zugesicherten) spenden begleichen zu müssen?
			* frage: war dafür nicht die rücklage für mietausfall gedacht?
				* nicht als ständige verpflichtung, nur im falle des falles zur kurzfristigen liquidation
				* anmerkung: mehr spenden zurücklegen, um rücklagen zu bilden? puffer über zeit, falls spendenaufkommen sich in einem jahr verringert.
			* anmerkung: was man sich nicht leisten kann, soll man sich nicht kaufen.
			* zum meinungsbild melden sich etwa vier leute. auch zur frage, nur übergangsweise spenden zu investieren, ähnliches meinungsbild.
		* anmerkung: aktueller standortvorteil wird vermutlich an keinem weiteren standort in braunschweig zusammentreffen.
		* anmerkung: meinungsbild ohne ein konkretes objekt im sinn ist schwer zu beurteilen.
			* für den verein aber immerhin ein wichtiger punkt bei der weiteren suche nach objekten. wollen wir objekte suchen, die wir uns aktuell nicht leisten können, oder nicht?
		* anmerkung: vor umzug in space2 wurde puffer von mitgliedern in form von vorauszahlung auf den mitgliedsbeitrag gebildet, zb für einmalige einrichtungsgegenstände
		* anmerkung: mitglieder, die sich das leisten könnten, könnten für einen voraussehbaren zeitraum ihren beitrag erhöhen.
		* anmerkung: wachstum ist im moment offenbar vorhanden (siehe finanzbericht), man könnte extrapolieren und entsprechend auf die zukunft planen.
		* anmerkung: zu spenden: falls eine organisation ihre großspende nicht mehr für die zukunft zusagt, fliegt man auf die füße. erhöhter mitgliedsbeitrag ist sinnvollere investition, da viele kleinere beträge, die nicht so schnell fluktieren sollten
		* anmerkung: wenn räumlichkeit erweitert werden soll (statt komplettumzug), kann man durchaus spenden investieren, weil man die zusätzliche räumlichkeit später wieder schnell abgestoßen werden kann.
		* anmerkung: man sollte wissen, wie viele spenden dafür akquiriert werden müssten. durch mehr erhöhte mitgliederbeiträge ist man auch nicht mehr so auf großspenden überwiesen.
		* anmerkung: beitrag erhöhen ohne ziel macht wenig sinn, sondern wird dann eher "allgemein" verwendet. lieber eine zusage für erhöhung zur erreichung eines definitiven ziels
		* meinungsbild wird nochmal gestellt: wollen wir uns nach räumlichkeiten auscchau halten, die unsere monatlichen mitgliedsbeiträge übersteigt? bei finanzierung zb durch freiwillige erhöhung der mitgliedsbeiträge oder großspenden? meinungsbild fällt diesmal positiv aus.
		* generell: ermutigung an alle mitglieder, bei der suche mitzumachen, vorstand ist eher für formalfoo zuständig und nicht dafür, alle TODOs zu übernehmen :)
			* beteiligung an vorstandssitzungen und arbeitstreffen sind auch sehr gern gesehen und ihr könnt auch eigene aspekte einbringen

%%%%%%%%%%%%%
%% TOP 4.2 %%
%%%%%%%%%%%%%
\subsection{Weiteres}

Folgende Punkte werden aufgrund der fortgeschrittenen Zeit vertagt:
\begin{itemize}
  \item Weltherrschaft
  \item Keysigning-Party
  \item Anerkennung des RaumZeitLabors als Außenstelle des Stratum~0
\end{itemize}

Dem scheidenden Vorstand wird für seine Arbeit gedankt. Die Anwesenden werden
dazu aufgefordert, bis zu den Vorträgen am Abend zu bleiben.
\meetingend{16:37}
\end{document}

% vim: set et ts=2 sw=2 sts=2 :
