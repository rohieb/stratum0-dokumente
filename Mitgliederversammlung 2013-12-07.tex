% vim:tw=80 ts=2 et sw=2 indentexpr= :
\documentclass[a4paper,12pt]{scrartcl}
\usepackage[utf8]{inputenc}
\usepackage[T1]{fontenc}
\usepackage[ngerman]{babel}
\usepackage{libertine} % kann man notfalls auch ignorieren, wenns nicht da ist
\usepackage{textcomp} % notfalls für €
\usepackage{stratum0doc}
\usepackage[colorlinks=false]{hyperref}
\usepackage{graphicx}
\usepackage[defaultmono,scale=0.8]{droidmono}
\urlstyle{tt}

\renewenvironment{quote}{%
  \list{}{\rightmargin\leftmargin}%
    \item\small\itshape\relax%
}{%
  \endlist
}


\title{4.~Mitgliederversammlung des Stratum~0~e.~V.}
\date{7. Dezember 2013}

\begin{document}
\maketitle
%{\footnotesize\tableofcontents}

%%%%%%%%%%%
%% TOP 0 %%
%%%%%%%%%%%
\section{Organisatorischer Overhead}
\begin{description}
  \item[Zeit:] Samstag, 7. Dezember 2013, 14:00
  \item[Ort:] neue Räumlichkeiten des Stratum 0 e.~V., Hamburger Straße 273a
  \item[Anwesend:] zu Beginn 18 von 61 Mitgliedern (29{,}5\,\%), 1 nicht
    akkreditiertes Mitglied. Die Versammlung ist somit beschlussfähig. Gäste
    werden ohne Gegenstimmen zugelassen, es sind aber zu Beginn noch keine
    anwesend.
  \item[Wahl des Versammlungsleiters:] Valodim einstimmig durch Handzeichen;
    nimmt die Wahl an
  \item[Protokoll:] rohieb einstimmig durch Handzeichen, nimmt die Wahl an.
  \item[Veranstaltung eröffnet] durch den Versammlungsleiter um 14:10
  \item[Tagesordnung:] ohne Gegenstimmen angenommen.
\end{description}

\paragraph{Anträge zur Geschäftsordnung} Ein Antrag, der es anwesenden
Mitgliedern erlauben würde, die Versammlung vor deren Ende zu verlassen und ihr
Stimmrecht auf andere anwesende Mitglieder zu übertragen, wird nach kurzer
Umfrage unter den Mitgliedern wieder zurückgezogen. Es werden keine weiteren
Anträge zur Geschäftsordnung gestellt.

\paragraph{Genehmigung des Protokolls der Mitgliederversammlung 2013-03-16} Das
Protokoll wird während der Versammlung unter den Mitgliedern herumgereicht. Es
wird jedoch bis zum Ende der Veranstaltung versäumt, darüber abzustimmen.


%%%%%%%%%%%
%% TOP 1 %%
%%%%%%%%%%%
\section{Jahresbericht und Entlastung des Vorstandes}
Stellvertretend für den Vorstand trägt der Vorstandsvorsitzende die Aktivitäten
des Vorstandes im vergangenen Jahr vor (siehe Präsentationsfolien im
Anhang~\ref{sec:jahresbericht}), anschließend fasst der Schatzmeister kurz die
Finanzen zusammen (siehe Anhang~\ref{sec:finanzen}). Schließlich präsentiert
reneger die Aktivitäten während des Umbaus zum Space 2.0 (siehe
Anhang \ref{sec:space2.0}).

\emph{Während der Präsentationen werden noch 8 verspätet eintreffende
  Mitglieder akkreditiert.}

Die Finanzen bewegen sich trotz der Renovierung der neuen Räumlichkeit in einem
stabilen Rahmen. Die Kassenprüfer haben noch einige kleinere
Verbesserungsvorschläge bezüglich der Buchführung, die sie mit dem Schatzmeister
nochmal besprechen werden. Die Buchführung ist schon deutlich
nachvollziehbarer als im letzten Jahr, es gibt Buchungsreferenzen und
nummerierte Vorstandsbeschlüsse. Kleinere Fehler sind noch da, aber das passiert
halt; grundsätzlich empfehlen die Kassenprüfer, vertreten durch shoragan, die
Entlastung des Vorstandes.

\vote{Entlastung des Vorstandes}{26}{0}{0}
Es wird über die Entlastung des Vorstandes per Handzeichen abgestimmt. Die
Abstimmung fällt einstimmig für die Entlastung des Vorstandes aus.

\emph{Es gibt eine kurze Pause von 15:45 bis 16:00.}

%%%%%%%%%%%
%% TOP 2 %%
%%%%%%%%%%%
\section{Wahlen}
\consensus{Kassenprüfer werden diesmal nicht gewählt}
Die Amtszeit der Kassenprüfer ist laut Satzung nicht beschränkt. Ein kurzes
Meinungsbild ergibt, dass niemand die Kassenprüfer neu wählen will; außerdem
würden beide Kassenprüfer weiterhin das Amt übernehmen. Eine Wahl der
Kassenprüfer entfällt somit.

Als wahlleitende Entität stellt sich ktrask zur Verfügung, die Mehrheit der
anwesenden Mitglieder hat kein Problem damit.

Die Stimmzettel wurden bei der Akkreditierung ausgegeben, es gibt je einen
Stimmzettel für die Wahl des Vorstandsvorsitzenden, des stellv. Vorsitzenden,
des Schatzmeisters und der Beisitzer, die Beisitzer werden gemeinsam gewählt.
Für alle Ämter wird in einem Wahlgang gestimmt.  Als Wahlverfahren wird die Wahl
durch Zustimmung vorgeschlagen: Jedes stimmberechtige Mitglied darf beliebig
viele Kandidaten pro Stimmzettel ankreuzen. Diejenigen Kandidaten, die die
meisten Stimmen und mindestens 50\,\% der abgegebenen Stimmen auf sich vereinen
können, sind gewählt, wobei jeder Kandidat nur ein Amt innehaben kann. Die Wahl
für die einzelnen Vorstandsämter wird dann nach folgender Reihenfolge
ausgewertet: Vostandsvorsitzender, stellv. Vorsitzender, Schatzmeister,
Beisitzer.

Die einzelnen Kandidaten stellen sich kurz vor.

Die Wahlleiterin eröffnet die Wahl um 16:19 und schließt die Wahl um 16:21. Als
Wahlhelfer melden sich whisp, Neo Bechstein und joke.

Das Ergebnis der Auszählung wird um 16:52 verkündet. Es wurden für alle
Vorstandsämter 25 Stimmzettel abgegeben, allerdings war für die Wahl des
Vorstandsvorsitzenden ein Stimmzettel ungültig, was aber nichts am Ergebnis
ändert.

Das Ergebnis lautet wie folgt:

\paragraph{Vorstandsvorsitzender}
\begin{itemize}
  \item Valodim: 19 Stimmen
  \item larsan: 14 Stimmen
  \item rohieb: 13 Stimmen
  \item chrissi\textasciicircum: 7 Stimmen
\end{itemize}

\elected{Vorstands\-vorsitzender}{Valodim}{19}{24}
Valodim nimmt die Wahl an und ist somit im Amt als Vorstandsvorsitzender
bestätigt.

\paragraph{Stellvertretender Vorsitzender}
\begin{itemize}
  \item larsan: 20 Stimmen
  \item rohieb: 19 Stimmen
  \item Valodim: 14 Stimmen
  \item chrissi\textasciicircum: 11 Stimmen
  \item Pecca: 15 Stimmen
\end{itemize}

\elected{Stellv. Vorsitzender}{rohieb}{19}{25}
larsan nimmt die Wahl nicht an. rohieb als Kandidat mit der nächsthöheren
Stimmenanzahl nimmt die Wahl an und ist somit im Amt als stellvertretender
Vorsitzender bestätigt.

\paragraph{Schatzmeister}
\begin{itemize}
  \item chrissi\textasciicircum: 24 Stimmen
  \item shoragan: 6 Stimmen
  \item rohieb: 5 Stimmen
  \item DooMMasteR: 1 Stimme
\end{itemize}

\elected{Schatzmeister}{chrissi\textasciicircum}{24}{25}
chrissi\textasciicircum{} hat die Veranstaltung zu diesem Zeitpunkt
schon verlassen. Er wird angerufen und erklärt die Annahme der Wahl telefonisch.

\paragraph{Beisitzer}
\begin{itemize}
  \item larsan: 24 Stimmen
  \item Pecca: 22 Stimmen
  \item rohieb: 22 Stimmen
  \item hellfyre: 19 Stimmen
  \item reneger: 18 Stimmen
  \item Valodim: 17 Stimmen
  \item Kasalehlia: 16 Stimmen
  \item comawill: 15 Stimmen
  \item shoragan: 11 Stimmen
  \item drc: 9 Stimmen
  \item Neo Bechstein: 4 Stimmen
  \item Terminar: (Kandidatur zurückgezogen)
\end{itemize}

\elected{Beisitzer}{larsan}{24}{25}
\elected{Beisitzerin}{Pecca}{22}{25}
\elected{Beisitzer}{hellfyre}{19}{25}
rohieb ist schon als stellvertretender Vorsitzender gewählt und entfällt damit.
larsan, Pecca und hellfyre nehmen die Wahl an.

Der neue Vorstand besteht somit aus Valodim (Vincent Breitmoser) als
Vorstandsvorsitzender, rohieb (Roland Hieber) als stellv. Vorsitzender,
chrissi\textasciicircum{} (Chris Fiege) als Schatzmeister sowie larsan (Lars
Andresen), Pecca (Rebecca Husemann) und hellfyre (Matthias Uschok) als
Beisitzer.

%%%%%%%%%%%
%% TOP 3 %%
%%%%%%%%%%%
\section{Änderungsanträge}

\emph{Es sind zu diesem Zeitpunkt 23 Mitglieder anwesend. Das Quorum von drei
Vierteln der anwesenden Mitglieder zur Änderung der Satzung liegt somit bei 18
Stimmen, das Quorum von der Hälfte der anwesenden Mitglieder für die Änderung
der Beitragsordnung bei 12 Mitgliedern.}

Die Änderungsanträge werden durch Handzeichen abgestimmt.

\subsection{Beitragsordnung: Lastschriftverfahren}
Es soll den Mitgliedern die Möglichkeit gegeben werden, den Beitrag auch per
Lastschrift einziehen zu lassen, um die Pünktlichkeit und den
Automatierungsgrad der Zahlungen zu erhöhen. Der betreffende §2 Abs.~1 der
Beitragsordnung lautet wie folgt:

\begin{quote}
  Die Zahlung des Beitrages erfolgt per Überweisung.
\end{quote}
Dieser Absatz soll auf den folgenden Wortlaut geändert werden:
\begin{quote}
  Die Zahlung des Mitgliedsbeitrages kann per Überweisung (z.~B.  Dauerauftrag)
  oder per SEPA-Lastschrifteinzug erfolgen. Für den Einzug per SEPA-Lastschrift
  muss dem Vorstand ein SEPA-Lastschriftmandat in Schriftform vorliegen.
  Eventuell anfallende Gebühren durch Rücklastschrift, die ein Mitglied selbst
  zu verschulden hat, werden dem Mitglied in Rechnung gestellt.
\end{quote}

\vote{Beitragsordnung: Lastschrift ermöglichen}{23}{0}{0}
Die Abstimmung per Handzeichen fällt einstimmig für den Antrag aus. Damit sind
in Zukunft Lastschriftzahlungen möglich.

\subsection{Einführung von Fördermitgliedschaften}

Die Einführung von Fördermitgliedschaften soll es natürlichen und juristischen
Personen einfacher machen, den Verein mit einem regelmäßigen Beitrag zu
unterstützen, ohne sich unbedingt einbringen zu müssen. Im Moment gibt es zwar
die Möglichkeit, einen beliebigen Beitrag nach der Härtefallregelung zu
beantragen. Der Vorstand hat in diesem Fall zwar immer einen geringeren Beitrag
für Mitglieder gewährt, jedoch geht es hier prinzipiell nur um Härtefälle, und
ein Beitragsmodell für Fördermitgliedschaften wurde schon mehrfach angefragt.

Die Satzung und die Beitragsordnung sollen wie folgt angepasst werden:

\begin{itemize}
  \item In §4 der Satzung soll der neue Abs.~2 eingefügt werden, der die
    Fördermitgliedschaft definiert:
    \begin{quote}
      Die Mitgliedschaft im Verein ist auf zwei Arten möglich:
    \begin{itemize}
      \item Ordentliche Mitglieder gestalten das Vereinsleben durch ihre aktive
        Teilnahme mit. Sie besitzen eine Stimmberechtigung auf den
        Mitgliederversammlungen des Vereins.
      \item Fördermitglieder unterstützen den Verein vorrangig durch ihren
        regelmäßigen finanziellen Beitrag. Sie besitzen keine Stimmberechtigung
        auf den Mitgliederversammlungen. 
    \end{itemize}
    \end{quote}
    Die Nummerierung der bisherigen Absätze 2 bis 5 wird entsprechend angepasst.

  \item §7 Abs.~6 Satz 4 der Satzung wird ergänzt und legt das Stimmrecht auf
    der Mitgliederversammlung fest. Der bisherige Text lautet: 
    \begin{quote}
      Jedes Mitglied hat eine Stimme.
    \end{quote}
    Dieser Satz wird wie folgt ergänzt: 
    \begin{quote}
      Jedes ordentliche Mitglied hat eine Stimme. Fördermitglieder sind
      berechtigt, an den Versammlungen ohne Stimmrecht teilzunehmen.
    \end{quote}

  \item §0 Abs.~1 der Beitragsordnung regelt die Beiträge für Fördermitglieder.
    Der bisherige Text lautet:
    \begin{quote}
      Der reguläre Mitgliedsbeitrag beträgt 20\,€ pro Monat.
    \end{quote}
    Der Absatz wird auf den folgenden Wortlaut geändert:
    \begin{quote}
      Der reguläre Mitgliedsbeitrag für ordentliche Mitglieder beträgt 20\,€
      pro Monat. Fördermitglieder zahlen einen frei wählbaren Beitrag von
      mindestens 30\,€ pro Jahr.
    \end{quote}

  \item §0 Abs.~2 Satz 1 der Beitragsordnung wird geändert, sodass der ermäßigte
    Beitrag nur für die ordentliche Mitgliedschaft gilt. Der bisherige Text
    lautet: 
    \begin{quote}
      Schüler, Studenten, Auszubildende, Empfänger von Sozialgeld oder
      Arbeitslosengeld II einschließlich Leistungen nach § 22 ohne Zuschläge
      oder nach § 24 des Zweiten Buchs des Sozialgesetzbuchs (SGB II), sowie
      Empfänger von Ausbildungsförderung nach dem
      Bundesausbildungsförderungsgesetz (BAföG) haben die Möglichkeit, einen
      ermäßigten Beitrag von 12\,€ pro Monat zu zahlen.
    \end{quote}
    Der Satz wird auf die folgende Formulierung geändert:
    \begin{quote}
      Schüler, Studenten, Auszubildende, Empfänger von Sozialgeld oder
      Arbeitslosengeld II einschließlich Leistungen nach § 22 ohne Zuschläge
      oder nach § 24 des Zweiten Buchs des Sozialgesetzbuchs (SGB II), sowie
      Empfänger von Ausbildungsförderung nach dem
      Bundesausbildungsförderungsgesetz (BAföG) haben die Möglichkeit, für die
      ordentliche Mitgliedschaft einen ermäßigten Beitrag von 12\,€ pro Monat zu
      zahlen.
    \end{quote}

  \item In §0 Abs.~3 Satz 1 der Beitragsordnung wird der Term "`Mitglied"' in
    "`ordentliches Mitglied"' geändert. Der bisherige Text lautet:
    \begin{quote}
      Sollte ein Mitglied aus finanziellen Gründen den Mitgliedsbeitrag
      nicht aufbringen können, kann dieses beim Vorstand einen Antrag auf
      Ermäßigung oder Befreiung stellen.
    \end{quote}
     Der Satz wird auf den folgenden Wortlaut geändert:
     \begin{quote}
        Sollte ein ordentliches Mitglied aus finanziellen Gründen den
        Mitgliedsbeitrag nicht aufbringen können, kann dieses beim Vorstand
        einen Antrag auf Ermäßigung oder Befreiung stellen.
     \end{quote}
\end{itemize}

Es gibt zwei Fragen zu diesem Antrag:

\question{Zählen Fördermitglieder mit zum Quorum von 23\,\% der Mitglieder, die
für eine beschlussfähige Mitgliederversammlung benötigt werden?}
\answer{Nein, Fördermitglieder können ja nicht abstimmen.}
\emph{Anmerkung: diese Frage wurde auf der Versammlung so beantwortet,
allerdings bestimmt §7 Abs.~6 der Satzung die Beschlussfähigkeit aufgrund der
Mitglieder und nicht aufgrund der stimmberechtigten Mitglieder, sodass 
Fördermitglieder durchaus zum Quorum dazuzählen.}

\question{Dürfen juristische Personen ordentliche Mitglieder werden?}
\answer{Darüber wird nichts ausgesagt, ist prinzipiell nicht verboten, der
Vorstand entscheidet darüber.}

\vote{Einführung Fördermitgliedschaften}{23}{0}{0}
Die Versammlung spricht sich einstimmig per Handzeichen dafür aus, den Antrag
anzunehmen. In Zukunft wird es also ein Beitragsmodell für Fördermitglieder
geben.

\subsection{Beitragsbefreiung für maximal ein Jahr}

In §0 Abs.~3 Satz 2 der Beitragsordnung soll das Wort "`maximal"' eingefügt
werden, um dem Vorstand mehr Flexibilität bei der Bewilligung von individuellen
Beiträgen für ordentliche Mitglieder nach der Härtefallregelung zu geben. Der
bisherige Text lautet:
\begin{quote}
  Diese gilt für ein Jahr und kann dann durch einen neuen Antrag erneuert
  werden.
\end{quote}
Der neue Text lautet:
\begin{quote}
  Diese gilt für maximal ein Jahr und kann dann durch einen neuen Antrag
  erneuert werden.
\end{quote}

\vote{Beitragsbefreiung für maximal ein Jahr}{23}0{0}
Es gibt keine Fragen zu diesem Antrag. Der Antrag wird einstimmig angenommen.

\subsection{Satzung: Beschlüsse der Mitgliederversammlung haben Vorrang}

§7 Abs.~7 der Satzung soll durch folgenden Satz ergänzt werden:
\begin{quote}
  Im Fall von gegensätzlichen Beschlüssen haben die Beschlüsse der
  Mitgliederversammlung Vorrang vor denen anderer Vereinsorgane.
\end{quote}

Der Antrag wurde gestellt, um Unsicherheiten in Bezug auf sich widersprechende
Entscheidungen des Vorstands und der Mitgliederversammlung in Zukunft zu
vermeiden.

Als Gegenargument für den Antrag wird angeführt, dass die Satzung in §7 Abs.~4
schon regelt, dass die Mitgliederversammlung das oberste beschlussfassende
Vereinsorgan ist.

\vote{Beschlüsse der Mitgliederversammlung haben Vorrang}{1}{12}{6}
Dementsprechend wird der Antrag mit 12 Contra-Stimmen gegen eine Pro-Stimme und
6 Enthaltungen abgelehnt.

\subsection{Satzung: obsoleten Absatz entfernen}

§8 Abs.~12 der Satzung soll entfernt werden, da die Regelung nicht mehr
angewendet wird. Der bisherige Text lautet:
\begin{quote}
  Die Amtszeit des auf der Gründungsversammlung gewählten Vorstandes endet mit
  der ersten Mitgliederversammlung.
\end{quote}

\vote{Satzung: obsoleten Absatz entfernen}{17}{2}{2}
Die Abstimmung ergibt 17 Stimmen für den Antrag, 2 Enthaltungen und 2
Gegenstimmen und hat damit nicht die nötige Dreiviertelmehrheit von 8 Stimmen
für Satzungsänderungen erreicht. Der Absatz bleibt also bestehen.

\begin{description}
  \item[Versammlung geschlossen] um 17:16
\end{description}


%%%%%%%%%%%%%
%% Anhänge %%
%%%%%%%%%%%%%
\appendix

%%%%%%%%%%%%%%
%% Anhang A %%
%%%%%%%%%%%%%%
\cleardoublepage
\section{Präsentation Jahresbericht 2013}\label{sec:jahresbericht}
\includepdfpage{images/2013-12-07-rueckblick-2013.pdf}{1}
\includepdfpage{images/2013-12-07-rueckblick-2013.pdf}{2}

\includepdfpage{images/2013-12-07-rueckblick-2013.pdf}{3}
\pdfpagenote{Mehr zu Space 2.0 von reneger in \ref{sec:space2.0}}
\includepdfpage{images/2013-12-07-rueckblick-2013.pdf}{4}
\pdfpagenote{Zahlen zum Crowdfunding in \ref{sec:finanzen}}

\includepdfpage{images/2013-12-07-rueckblick-2013.pdf}{5}
\pdfpagenote{Beim Multikoptertreffen etwa so einmal im Monat ein neues Gesicht.
  Mangel an richtiger Werkstatt mit mehr als einem Lötkolben, Frickelraum ist
  leider dauerbelegt. Game-Development-Runde trifft sich unregelmäßig, aber
  laufend. Leuchtkramtreffen noch neu, eher Richtung Kunst und nicht nur
  Hacking.}
\includepdfpage{images/2013-12-07-rueckblick-2013.pdf}{6}

\includepdfpage{images/2013-12-07-rueckblick-2013.pdf}{7}
\pdfpagenote{Blog: Reichweite schwer zu schätzen, da keine Statistiken; geteilte
  Meinungen über die Blogsoftware. Außenwirkung der Mailingliste ist eher
  "`amüsierend"'. Dedizierter Newsletter für Ankündigungen und Termine.}
\includepdfpage{images/2013-12-07-rueckblick-2013.pdf}{8}
\pdfpagenote{Mehrere Admin-Treffen zur Infrastrukturplanung. Dienste laufen
  hauptsächlich bei shoragan (DynDNS stratum0.net) bzw. Neo Bechstein und
  lichtfeind (Mail, Wiki). Regelmäßige Backups sind ausreichend vorhanden.}

\includepdfpage{images/2013-12-07-rueckblick-2013.pdf}{9}
\pdfpagenote{Pressespiegel siehe \url{https://stratum0.org/wiki/Pressespiegel}.
  Kryptoparty mit etwa 10 externen Interessenten. Vortrag beim Studium Generale
  über 3D-Druck, etwa 200 Leute im Audimax der TU. Zeromutarts CTF für
  Informatik-Erstis zur Mitgliederwerbung. Außerdem Reparaturworkshop "`Fixed
  und fertig"' an der HBK. Fazit: Gute Außenwahrnehmung.}
\includepdfpage{images/2013-12-07-rueckblick-2013.pdf}{10}
\pdfpagenote{Anime-Abend war auch durch die Raumgröße beschränkt.}

\includepdfpage{images/2013-12-07-rueckblick-2013.pdf}{11}
\pdfpagenote{Auf Vorstandssitzungen hauptsächlich Formalitäten; Geschäftsordnung
  siehe \url{https://stratum0.org/wiki/Gesch\%C3\%A4ftsordnung_Vorstand}}
\includepdfpage{images/2013-12-07-rueckblick-2013.pdf}{12}
\pdfpagenote{GEZ: 6\,€ pro Monat, fällt weg bei Gemeinnützigkeit.}

\includepdfpage{images/2013-12-07-rueckblick-2013.pdf}{13}
\pdfpagenote{Fablab-Vorstand hat eine Firma gegründet, Verein hat seitdem
  anscheinend an Schwung verloren.}
\includepdfpage{images/2013-12-07-rueckblick-2013.pdf}{14}
\pdfpagenote{Maker Faire: Kommunikation nicht optimal, jemand hatte einen Stand
  angemeldet, aber niemand hatte Zeit. Gemeinnützigkeit hängt von bereinigten
  Büchern ab, Schatzmeister hat diese für die erste Steuererklärung
  aufgearbeitet.  Vorstand ist sich bewusst, dass das ein drängendes Thema ist.
  Steuerberater wird vorgeschlagen, aber der kostet halt auch was. Plenum hatte
  außer Aufräumerinnerungen keine Themen. Aufräumsituation könnte sich mit mehr
  Platz nach dem Umzug bessern, bleibt abzuwarten.}

\includepdfpage{images/2013-12-07-rueckblick-2013.pdf}{15}
\pdfpagenote{Positives: Yay, wir leben noch! Raumqualität ist offenbar über dem
  Durchschnitt anderer Hackerspaces. Projekte wurden früher eher schnell liegen
  gelassen, das hat sich gebessert. Feedback zur Vorstandsarbeit fehlt, aber auf
  jeden Fall auch keine negativen Stimmen. Verein ist finanziell gesund, nicht
  zu viel Geld wie letztes Jahr, aber auch nicht zu wenig.}
\includepdfpage{images/2013-12-07-rueckblick-2013.pdf}{16}


%%%%%%%%%%%%%%
%% Anhang B %%
%%%%%%%%%%%%%%
\cleardoublepage
\section{Präsentation Finanzen 2013}\label{sec:finanzen}
\includepdfpage{images/2013-12-07-jahresbericht-schatzmeister.pdf}{1}
\includepdfpage{images/2013-12-07-jahresbericht-schatzmeister.pdf}{2}

\includepdfpage{images/2013-12-07-jahresbericht-schatzmeister.pdf}{3}
\includepdfpage{images/2013-12-07-jahresbericht-schatzmeister.pdf}{4}

\includepdfpage{images/2013-12-07-jahresbericht-schatzmeister.pdf}{5}
\pdfpagenote{Spenden: 100\,€/Monat von Sponsoren, 2{.}500\,€ aus Spendentöpfen
  (inkl. T-Shirts/Hoodies)}
\includepdfpage{images/2013-12-07-jahresbericht-schatzmeister.pdf}{6}
\pdfpagenote{CTF-Gewinn ist direkt für den neuen Beamer geflossen. Allgemeine
  Ausgaben: nichts Space-relevantes, hauptsächlich aber Einkauf der
  T-Shirts/Hoodies.}

\includepdfpage{images/2013-12-07-jahresbericht-schatzmeister.pdf}{7}
\pdfpagenote{Miete/Nebenkosten: inkl. Internet. Einrichtung: inkl.
  Stickmaschine.  Renovierung Space 2.0 unter Vorbehalt, da noch nicht
  abgeschlossen.}
\includepdfpage{images/2013-12-07-jahresbericht-schatzmeister.pdf}{8}
\pdfpagenote{50\,€ für einmalige Verpflegung während eines CTF-Events.}

\includepdfpage{images/2013-12-07-jahresbericht-schatzmeister.pdf}{9}
\pdfpagenote{Einnahmen Bastelmaterial: hauptsächlich Spenden für 3D-Druck. Im
  Moment wird Filament für 3D-Druck zur Hälfte aus diesem Spendentopf, zur
  anderen Hälfte vom Verein bezahlt; diese Regelung könnte man überdenken, da
  genug Spenden reinkommen.}
\includepdfpage{images/2013-12-07-jahresbericht-schatzmeister.pdf}{10}
\pdfpagenote{Durch einen Fehler in der Buchhaltung gibt es mehr Schlüssel als
  zurückgelegtes Pfandgeld, handelt sich aber um kleine Beträge und wird auf
  jeden Fall zurückgezahlt.}

\includepdfpage{images/2013-12-07-jahresbericht-schatzmeister.pdf}{11}
\includepdfpage{images/2013-12-07-jahresbericht-schatzmeister.pdf}{12}
\pdfpagenote{Insgesamt 1{.}700\,€ Gewinn im letzten Jahr, trotz Umbau.}

\includepdfpage{images/2013-12-07-jahresbericht-schatzmeister.pdf}{13}
\pdfpagenote{Größere Einschnitte: Crowdfunding Stickmaschine, Holodeck, Kappsäge}
\includepdfpage{images/2013-12-07-jahresbericht-schatzmeister.pdf}{14}
\pdfpagenote{Mysteriöser Anstieg der Einnahmen vor der letzten
  Mitgliederversammlung im März\ldots}

\includepdfpage{images/2013-12-07-jahresbericht-schatzmeister.pdf}{15}
\pdfpagenote{Größere sonstige Ausgaben: Beamer (Dezember 2012), 
  T-Shirts/Hoodies (März 2013), Stickmaschine (Juli 2013), Erweiterung
  3D-Drucker (September 2013)}
\includepdfpage{images/2013-12-07-jahresbericht-schatzmeister.pdf}{16}
\pdfpagenote{Stand: heute morgen.}

\includepdfpage{images/2013-12-07-jahresbericht-schatzmeister.pdf}{17}
\pdfpagenote{Rückstellung für die Erhöhung der Mietsicherheit im neuen Space.
  Rücklage für Einnahmeausfälle falls Mitgliederentwicklung negativ, deckt aber
  nicht die gesamten 5 Jahre Mindestmietdauer ab; eher eine Zwischenlösung, weil
  6{.}000\,€ ungenutzes Kapital (auch für das Finanzamt) wohl kaum vertretbar
  sind.}
\includepdfpage{images/2013-12-07-jahresbericht-schatzmeister.pdf}{18}

\includepdfpage{images/2013-12-07-jahresbericht-schatzmeister.pdf}{19}
\pdfpagenote{Fazit: Verein lebt noch und ist flüssig, kann aber keine allzu
  hohen Sprünge mehr machen.

\question{Verhältnis von geforderten zu gezahlten Mitgliedsbeiträgen?}
\answer{hat der Schatzmeister gerade nicht zur Hand.}
}

\includepdfpage{images/2013-12-07-jahresbericht-schatzmeister.pdf}{20}
\pdfpagenote{Schatzmeister bedankt sich für das Vertrauen und hat den Job gern
  gemacht.}

% letzte Seite ist nur Abspann
%\includepdfpage{images/2013-12-07-jahresbericht-schatzmeister.pdf}{21}

%%%%%%%%%%%%%%
%% Anhang C %%
%%%%%%%%%%%%%%
\cleardoublepage
\section{Präsentation Umbau Space 2.0}\label{sec:space2.0}
\includepdfpage{images/2013-12-07-space2-0.pdf}{1}
\includepdfpage{images/2013-12-07-space2-0.pdf}{2}

\includepdfpage{images/2013-12-07-space2-0.pdf}{3}
\includepdfpage{images/2013-12-07-space2-0.pdf}{4}
\enlargethispage{\baselineskip}
\pdfpagenote{"`Odyssee Nowo"': Verhandlungen seit März 2013, Ansprechpartner hat
  immer wieder gewechselt, Hälfte der Absprachen wurden seitens Nowo nicht
  eingehalten, unerklärliche Mahnungen ohne Grundlage. Daher Verzögerung des
  vertraglichen Einzugstermins vor drei Monaten, offizieller Mietbeginn ist
  jetzt 1. Januar 2014. Inzwischen neue Hausverwaltung im Schimmel-Hof (Bahr
  Immobilien aus Peine)
  
  Planung von der letzten Mitgliederversammlung: alles selber machen.
  Vermieterin und Nowo waren dagegen, wollten bessere Qualität haben, daher mit
  Architekt und Nowo Preisverhandlungen auf etwa 600\,€ Warmmiete/Monat, dafür
  keine Fixkosten.}

\includepdfpage{images/2013-12-07-space2-0.pdf}{5}
\includepdfpage{images/2013-12-07-space2-0.pdf}{6}
\pdfpagenote{Malerarbeiten auch von uns selbst vorgenommen.}

\includepdfpage{images/2013-12-07-space2-0.pdf}{7}
\pdfpagenote{Fußboden muss demnächst noch von uns gelegt werden, um den Rest
  kümmert sich Bahr Immobilien.}
\includepdfpage{images/2013-12-07-space2-0.pdf}{8}
\pdfpagenote{Dinge, die noch von uns erledigt werden müssen/wollen.}

\includepdfpage{images/2013-12-07-space2-0.pdf}{9}
\pdfpagenote{Außerdem: restliche Dinge aus dem alten Space nach unten tragen,
  dort oben klar Schiff machen und Wände streichen. Einweihung im Space 2.0 ist
  in einer Woche.}
\includepdfpage{images/2013-12-07-space2-0.pdf}{10}

%%%%%%%%%%%%%%%%%%%%
%% Unterschriften %%
%%%%%%%%%%%%%%%%%%%%
\cleardoublepage
\section*{Unterschriften}
\vspace{0.7cm}
\noindent Protokollführer: \hrulefill\hfill\phantom{c}\par
\vspace{0.7cm}
\noindent Vorstandsvorsitzender: \hrulefill\hfill\phantom{c}\par
\vspace{0.7cm}
\noindent Stellv. Vorsitzender: \hrulefill\hfill\phantom{c}\par
\vspace{0.7cm}
\noindent Schatzmeister: \hrulefill\hfill\phantom{c}\par
\vspace{0.7cm}
\noindent Beisitzerin: \hrulefill\hfill\phantom{c}\par
\vspace{0.7cm}
\noindent Beisitzer: \hrulefill\hfill\phantom{c}\par
\vspace{0.7cm}
\noindent Beisitzer: \hrulefill\hfill\phantom{c}\par
%\vspace{0.7cm}
%\noindent Beisitzer: \hrulefill\hfill\phantom{c}\par

\end{document}

