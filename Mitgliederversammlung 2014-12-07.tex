% vim:tw=80 ts=2 et sw=2 indentexpr= :
\documentclass[a4paper,12pt]{scrartcl}
\usepackage[utf8]{inputenc}
\usepackage[T1]{fontenc}
\usepackage[ngerman]{babel}
\usepackage{libertine} % kann man notfalls auch ignorieren, wenns nicht da ist
\usepackage{textcomp} % notfalls für €
\usepackage{hyperref}
\usepackage{stratum0doc}
\usepackage{longtable}

\title{5.~Mitgliederversammlung des Stratum~0~e.~V.}
\date{7. Dezember 2014}

\begin{document}
\maketitle
%{\footnotesize\tableofcontents}

\begin{description}
  \item[Zeit:] Samstag, 7. Dezember 2014, 14:00
  \item[Ort:] Räumlichkeiten des Stratum 0 e.~V., Hamburger Straße 273a
  \item[Anwesend:] 26 von 70 Mitgliedern (39\,\%), die Versammlung
    ist somit beschlussfähig. 2 anwesende Gäste werden ohne Gegenstimmen
    zugelassen.
  \item[Wahl des Versammlungsleiters:] Valodim, einstimmig durch Handzeichen
  \item[Protokoll:] rohieb, einstimmig durch Handzeichen
  \item[Veranstaltung eröffnet] durch den Versammlungsleiter um 14:16
  \item[Tagesordnung:] ohne Gegenstimmen angenommen
\end{description}


%%%%%%%%%%%
%% TOP 0 %%
%%%%%%%%%%%
\section{Organisatorischer Overhead}

\paragraph{Genehmigung des Protokolls der Mitgliederversammlung 2013-12-07}
Das Protokoll des letzten Mitgliederversammlung wird nach Abstimmung durch
Handzeichen mit eindeutiger Mehrheit angenommen.

%%%%%%%%%%%
%% TOP 1 %%
%%%%%%%%%%%
\section{Jahresbericht und Entlastung des Vorstandes}
\subsection*{Jahresbericht}
larsan fasst stellvertretend für den Vorstand das letzte Jahr aus der Sicht des
Vereins kurz zusammen, seine Präsentation ist auf der Homepage zu finden.
\footnote{\url{https://stratum0.org/wiki/Datei:Jahresbericht2014.pdf}}

\paragraph{Was war}
\hspace{-2ex}\footnote{siehe auch \url{https://stratum0.org/wiki/Timeline}}
\begin{description}
  \item[07.12.2013] Letzte Mitgliederversammlung
  \item[14.12.2013] Einweihungsfeier im Space 2.0 mit Vorträgen und Party.
    Diverse Hackerspaces haben Abordnungen gesandt, es ist voll.
  \item[17.12.2013] Holzanlieferung für Boden und Holodeck
  \item[26.--30.12.2013] Stratum 0-Assembly auf dem 30. Chaos Communication
    Congress
  \item[08.01.2014] Erstes Treffen mit dem Netz39 e.\,V. zur Bewerbung für das
    EasterHegg 2015
  \item[29.01.2014] Übergabe des Space 1.0 an die Hausverwaltung, nach
    Überführung in den Ausgangszustand
  \item[09.01.2014] Inbetriebnahme des Zugangssystems "`StratumKey"'
  \item[13--25.01.2014] Holz- und Ölarbeiten am Fußboden
  \item[09.02.2014] Die Küche steht wieder
  \item[21.04.2014] EasterHegg in Stuttgart, Netz39 und Stratum~0 bekommen den
    Zuschlag für das EasterHegg 2015 in Braunschwieg
  \item[17.06.2014] Kickoff für Freifunk Braunschweig
  \item[19.07.2014] Erstes Braunschweiger CoderDojo
  \item[25.--27.07.2014] Bau des Holodecks
  \item[23.08.2014] Spontanaktion "`Neues Außenschild"'
  \item[11.--12.10.2014] Werkstatttische und Bodenarbeiten am Holodeck
  \item[17.10.2014] Treppe zur zweiten Ebene des Holodecks
  \item[21.10.2014] Klappe zur zweiten Ebene des Holodecks 
\end{description}

\paragraph{Projekte}\hspace{-2ex}\footnote{siehe gleichnamige Seiten im Wiki
unter \url{https://stratum0.org}}
\begin{itemize}
  \item Holodeck: Zusätzlicher Lagerraum über der Werkstatt als Holzkonstruktion
  \item Freifunk: freies, kostenloses, dezentrales WLAN in Bürgerhand, steigende
    Anzahl Nutzer und möglicherweise Kooperation mit der Stadt Braunschweig
  \item CoderDojo: spielerische Einführung in die Programmierung für Kinder,
    bisher 5 Termine mit viel Erfolg
  \item Mensadisplay: jetzt mit Netzwerkanbindung, C- und Python-API,
    VLC-Anbindung sowie X-Server-Emulation; reist mit uns auf Veranstaltungen
  \item CTF-Team: war auf der DEFCON in Las Vegas, in Frankreich, in der
    Schweiz, hat aktuell den 4. Platz auf CTFtime und richtet das 31c3-CTF
    aus
  \item Vegan Academy: veganes Kochen, mehrmals pro Monat
  \item Anime Referat: gemeinsame Studie japanischer Bewegtbildern,
    zweiwöchentlich
  \item Vorträge: immer mal wieder\ldots
  \item Multikopter: regelmäßige Treffen im Sommer, klagen über fehlende
    Werkstatt
  \item OpenStreetMap-Stammtisch: seit März, unregelmäßig
\end{itemize}

\paragraph{Anschaffungen und Spenden}
\begin{itemize}
  \item Spende CNC-Fräse vom Institut für Photogrammetrie der TU Braunschweig
  \item Spende Drehbank und Standbohrmaschine, inkl. Werkzeuge
  \item Spende Schneidplotter inkl. Klebefolien
  \item Kappsäge
  \item Spende Regeltrenntrafo bis 250V
  \item Vereinshaftpflichtversicherung ab 22. Dezember 2014
\end{itemize}

\paragraph{Infrastuktur und Netzdienste}\hspace{-2ex}\footnote{siehe
\url{https://stratum0.org/wiki/Netzdienste}}
\begin{itemize}
  \item Wiki/Homepage: jetzt mit Kalender, könnte ein neues Design vertragen
  \item Blog: für Überdauerndes, gerne mehr Inhalte!
  \item Mailinglisten: Normalverteiler mit 170 Abonnenten, etwa 35 Mails pro
    Woche
  \item DynDNS (stratum0.net): jetzt auch mit pyongyanghackerspace.org
  \item IPv6 im Space funktioniert wieder
  \item Server: nach wie vor alle Webdienste auf privaten Maschinen gehostet,
    mehrfach redundante Backups. Eigener Server hoffentlich 2015, Wiki-Maschine
    stößt schon an ihre Grenzen\ldots
  \item WLAN: jetzt auch mit 802.11ac
  \end{itemize}

\paragraph{Zeitabgleiche -- Veranstaltungen mit unserer Beteiligung}
\begin{itemize}
  \item 30c3: 27.--30. Dezember, wenig Platz (für uns), trotz Umzug nach Hamburg
  \item GPN14: 19.--22. Juni in Karlsruhe
  \item Hackover: 24.--26. Oktober in Hannover, mit Mensadisplay!
  \item BarCamp Braunschweig: 28.--30. November, mit spontanem Hackcenter
\end{itemize}

\paragraph{Sonstiges}
\begin{itemize}
  \item Die Öffnungszeit nähert sich der 100\%-Marke
    \footnote{siehe \url{http://stratum0.wlnbrg.de/stats/stats.html}}
  \item Das Finanzamt Braunschweig hat dem Verein rückwirkend für 2014 die
    Gemeinnützigkeit zugesprochen, Spenden und Mitgliedsbeiträge können jetzt
    von der Steuer abgesetzt werden.
\end{itemize}

\paragraph{Was wird}
\begin{itemize}
  \item 31c3: Ende Dezember, hoffentlich mehr Platz, diesmal mit Mensadisplay!
  \item EasterHegg: Ostern 2015 in Braunschweig, zusammen mit dem Magdeburger
    Netz39 e.\,V. (siehe auch \ref{sec:eh15})
  \item Chaos Communication Camp, Sommer 2015
  \item MakerFaire Hannover
\end{itemize}

\paragraph{Was hoffentlich wird}
\begin{itemize}
  \item mehr Datenvisualisierung: Finanzen, Energie, Temperatur, Freifunk, etc.
  \item mehr Mitglieder
  \item Initiative ergreifen
  \item Offene Bauprojekte: \footnote{siehe
    \url{https://pad.stratum0.org/p/spacetodo}} Holodeck, Werkstatt, Leinwand
\end{itemize}

\subsection*{Finanzbericht}
Der Finanzbericht wird von chrissi\textasciicircum{}
vorgetragen, die Präsentation ist auf der Homepage zu finden.
\footnote{\url{https://stratum0.org/wiki/Datei:Finanzbericht_2014.pdf}}

\paragraph{Allgemeines}
Seit der letzten Mitgliederversammlung gibt es einige Veränderungen in der
Buchhaltung. Die Mate-Kasse ist nun Teil der Vereinsfinanzen und wird nicht mehr
privat geführt. Die Freigrenze der Einnahmen aus Wirtschaftlichem
Geschäftsbetrieb für die Körperschaftssteuer liegt hier bei 35{.}000\,€, die
Freigrenze für die Umsatzsteuer bei 17{.}500\,€.

Zur bequemen Erstattung von Rechnungen für Verbrauchsmaterial (Spülmittel,
Toilettenpapier, etc.) wurde eine offene Kasse eingerichtet, über welche
die Mitglieder sich die ausgelegten Beträge selbst erstatten können. Die
Rechnung muss dabei in der Kasse deponiert werden und wird wie üblich vom
Schatzmeister in die Buchhaltung aufgenommen. Der Schatzmeister füllt dabei die
Kasse auf maximal 20\,€ auf, um den Schaden im Falle eines Verlustes gering zu
halten.

Außerdem gibt es im Internet nun eine automatisch generierte Finanzübersicht der
Vereinsfinanzen\footnote{\url{https://stratum0.org/finanzuebersicht}}, ebenso
werden automatische Übersichten der Mitgliedskonten erzeugt und per E-Mail
verschickt.

\paragraph{Überblick}
Der Überblick über die Vereinsfinanzen vom 01.12.2013 bis 30.11.2014 zeigt die
größeren Kostenpunkte und Einnahmequellen:

\enlargethispage{1\baselineskip}
\begin{longtable}{|l|r|r|}
  \hline
  \textbf{Bereich} & \textbf{Einnahmen [€]} & \textbf{Ausgaben [€]} \\
  \hline
  \endfirsthead
  \multicolumn{3}{l}{\emph{(Fortsetzung von vorheriger Seite)}} \\
  \hline
  \textbf{Bereich} & \textbf{Einnahmen [€]} & \textbf{Ausgaben [€]} \\
  \hline
  \endhead
  \multicolumn{3}{l}{\emph{(Fortsetzung auf nächster Seite)}} \\
  \endfoot
  \endlastfoot
  Ideeller Bereich: Allgemein       & 13{.}622{,}63 &      -295{,}89 \\
  \quad davon Mitgliedsbeiträge     & 11{.}016{,}00 &                \\
  \quad davon Spenden               &  2{.}121{,}63 &                \\
  \quad davon Crowdfunding          &      485{,}00 &                \\
  \quad davon Kontoführungsgebühren &               &      -108{,}42 \\
  \hline
  Ideeller Bereich: Projekte        &      605{,}45 &      -794{,}32 \\
  \quad davon Bastelmaterial        &       30{,}24 &      -201{,}85 \\
  \quad davon Stickmaschine         &      173{,}79 &       -44{,}40 \\
  \quad davon 3D-Drucker            &       71{,}20 &      -264{,}64 \\
  \quad davon Freifunk              &      266{,}66 &      -112{,}08 \\
  \hline
  Ideeller Bereich: Space           &       34{,}65 & -14{.}873{,}27 \\
  \quad davon Miete und Nebenkosten &        4{,}65 &  -9{.}624{,}63 \\
  \quad davon Einrichtung           &       30{,}00 &  -1{.}905{,}46 \\
  \quad davon Renovierung Space 2.0 &               &  -3{.}155{,}14 \\
  \hline
  Zweckbetriebe                     &  1{.}330{,}54 &      -948{,}93 \\
  \quad davon Einweihung Space 2.0  &  1{.}130{,}54 &      -948{,}93 \\
  \quad davon Preisgeld CTF-Teilnahme &    200{,}00 &                \\
  \hline
  Wirtschaftlicher Geschäftsbetrieb &  5{.}424{,}45 &  -4{.}430{,}00 \\
  \quad (Matekasse) && \\
  \hline
  Mankobuchungen                    &       30{,}51 &        -1{,}80 \\
  \hline
  Gesamt:                           & 21{.}048{,}23 & -21{.}344{,}21 \\
  \hline\hline
  \textbf{Gewinn/Verlust Gesamt:}   &               &      -295{,}98 \\
  \hline
\end{longtable}
\vspace{-1\baselineskip}

Das Vereinsvermögen beläuft sich mit Stand vom 30.11.2014 insgesamt auf
6{.}289{,}04\,€.

Die Gegenüberstellung der durchschnittlichen Einnahmen und laufenden
Verpflichtungen pro Monat\footnote{siehe auch
\url{https://stratum0.org/wiki/Verträge}} zeigt ein positives Bild:

\begin{longtable}{|lr|lr|} % tabular kann keine \footnotes
  \hline
  \multicolumn{2}{|c|}{\textbf{Einnahmen}} &
  \multicolumn{2}{|c|}{\textbf{Verpflichtungen}\footnote{noch nicht
  berücksichtigt: Vereinshaftpflichtversicherung, 12{,}05\,€/Monat}} \\
  \hline
  Mitgliedsbeiträge: &     918\,€ & Miete, Nebenkosten:    & 630\,€ \\
  Spenden:           &     176\,€ & Strom:                 & 220\,€ \\
                     &            & Internet:              &  42\,€ \\
  \hline\hline
  Gesamt:            & 1{.}094\,€ & Gesamt:                & 892\,€ \\
  \hline
\end{longtable}

Die Einnahmen durch Mitgliedsbeiträge fluktuieren pro Monat stark je nach
Zahlungsrhythmus der einzelnen Mitglieder, bleiben im 3-Monats-Mittel jedoch
stabil. Die Nebenkosten durch Strom steigen an (siehe dazu auch
\ref{sec:stromanbieter}); Kosten durch Heizung sind im Moment noch nicht
absehbar, da seit dem Umzug noch keine Ablesung vorgenommen wurde.

Der Nebenkostenabrechnung 2013 (noch von nowo) wurde widersprochen, dort ergaben
sich Unstimmigkeiten bezüglich des gezahlten Abschlags. Bisher gab es noch keine
Reaktion der Hausverwaltung, andere Mieter haben auch von Problemen bei der
Nebenkostenabrechnung berichtet. Insgesamt geht es in unserem Fall aber um
weniger als 100\,€.

Ausstehend ist außerdem noch eine Förderung seitens der Volkswagen~AG, die sich
bereit erklärt hat, für die Reisekosten des CTF-Teams zur DEFCON in Las Vegas
aufzukommen.

\paragraph{Rücklagen}\label{sec:finanzbericht-ruecklagen}
Vom Vereinsvermögen sind mehrere Rückstellungen gebildet worden:

\begin{center}
\begin{tabular}{|l|r|}
  \hline
  \textbf{Zweck der Rücklage} & \textbf{Betrag} \\
  \hline
  Erhöhung Mietsicherheit Space 2.0 &     160{,}00\,€ \\
  Nachforderung naturstrom          &     700{,}00\,€\\
                                    &    + 75{,}00\,€/Monat \\
  Puffer Einnahmeausfälle           & 3{.}000{,}00\,€ \\
  \hline
\end{tabular}
\end{center}

Die Erhöhung der Mietsicherheit von 900\,€ auf 1060\,€ war im Mietvertrag zum
Space 2.0 geregelt, wurde aber vom Vermieter noch nicht eingefordert. Des
weiteren ist der Stromverbrauch seit dem Umzug über das erwartete Maß hinaus
angestiegen (auf Kosten von etwa 240\,€/Monat), jedoch wurde der Abschlag von
der naturstrom AG nicht angepasst, sodass eine Rücklage für die erwarteten
Nachforderungen bis Ende Dezember 2014 angelegt wurde, die im Moment pro Monat
um 75\,€ erhöht wird. Über die Rücklage zum Abfangen von Einnahmeausfällen wird
schließlich in \ref{sec:ruecklagen} entschieden.

Insgesamt ist die finanzielle Lage trotz Umzug in die neuen Räumlichkeiten und
einem geringen Verlust von etwa 300\,€ als stabil zu bewerten.

\paragraph{Mitgliederzahl}
Die Mitgliederzahl ist dabei seit der letzten Mitgliederversammlung von 61 auf
66 angestiegen, was vor allem durch das erweiterte Platzangebot zu erklären ist,
von dem mehrere Projekte profitierten. Nichtsdestotrotz sollte die
Nachwuchsarbeit nicht vernachlässigt werden.

\subsection*{Bericht der Kassenprüfer}
Die Kassenprüfer, vertreten durch shoragan, haben wenig an der Geschäftsführung
des Schatzmeisters auszusetzen. Die Belege waren vollständiger als letztes Jahr
und besser mit der Buchhaltung verknüpft. Es waren wenige Unstimmigkeiten
vorhanden, die aber alle nachvollzogen und ausgeräumt werden konnten. Insgesamt
macht die Buchhaltung einen ordentlichen Eindruck, und die Kassenprüfer
empfehlen die Entlastung des Schatzmeisters.

\question{Wirken sich die Umstimmigkeiten auf die Gemeinnützigkeit aus?}
\answer{Das ist nicht der Fall.}

\subsection*{Entlastung des Vorstandes}
\emph{Ein Mitglied kommt dazu und wird akkreditiert.}

Es wird vorgeschlagen, den Vorstand als Ganzes zu entlasten. Niemand fordert die
separate Entlastung einzelner Vorstandsmitglieder.

\vote{Entlastung des Vorstandes}{27}{0}{0}
Über die Entlastung des Vorstandes wird per Handzeichen abgestimmt. Die
Abstimmung fällt einstimmig für die Entlastung des Vorstandes aus.


%%%%%%%%%%%
%% TOP 2 %%
%%%%%%%%%%%
\section{Wahlen}

Die Kandidaten für die einzelnen Vorstandsposten wurden im Wiki
nominiert\footnote{siehe
\url{https://stratum0.org/wiki/Mitgliederversammlung_2014-12-07}} und stellen
sich jeweils selbst kurz vor.

Als Wahlleitung wird whisp gewählt. Als Wahlhelfer melden sich wolpertwo,
ktrask, bw\_, mkalte und Neo Bechstein; es gibt keine Einwände gegen die
Wahlhelfer.

\emph{Pause von 20 Minuten zur Vorbereitung der Wahl bis 16:30}

Nach der Pause wird der Wahlmodus bekannt gegeben. Die Wahl der einzelnen
Vorstandsposten erfolgt per Zustimmung (bei jedem der Kandidierenden kann bis zu
ein Kreuz gesetzt werden). Die einzelnen Posten werden in der Reihenfolge
Vorstandsvorsitzender, stellv.~Vorsitzender, Schatzmeister, Beisitzende besetzt,
schon gewählte Personen scheiden automatisch für andere Ämter aus. Die
Beisitzerposten werden absteigend nach Anzahl der erhaltenen Stimmen besetzt,
solange die Anzahl der erhaltenen Stimmen mehr als 50\% der abgegebenen Stimmen
ausmacht. Wenn kein eindeutiges Ergebnis ermittelt werden kann, wird eine
Stichwahl vorgenommen. Es gibt keine Einwände zu diesem Wahlmodus.

Das Ergebnis wird um 17:46 bekannt gegeben. Es wurden 26 Stimmzettel abgegeben,
alle davon waren gültig. Das Quorum für mehr als 50\% der abgegebenen Stimmen
liegt damit bei 14 Stimmen.

\subsection*{Vorstandsvorsitzender}
\begin{tabular}{|l|l|l|}
  \hline
  \textbf{Kandidat} & \textbf{Stimmen} & \textbf{Prozent} \\ \hline
  larsan (Lars Andresen) & 25 & 96.1\% \\
  Enthaltung             &  1 &  3.8\% \\
  \hline
\end{tabular}

\elected{Vorstands\-vorsitzender}{larsan}{25}{26}
larsan nimmt die Wahl als Vorstandsvorsitzender an.

\subsection*{Stellvertretender Vorsitzender}
\begin{tabular}{|l|l|l|}
  \hline
  \textbf{Kandidat} & \textbf{Stimmen} & \textbf{Prozent} \\ \hline
  rohieb     (Roland Hieber)  & 19 & 73.1\% \\
  Kasalehlia (Hilko Boekhoff) & 16 & 61.5\% \\
  reneger    (René Stegmaier) & 10 & 38.4\% \\
  \hline
\end{tabular}

\elected{Stellv. Vorsitzender}{rohieb}{19}{26}
rohieb nimmt die Wahl zum stellvertretenden Vorsitzen an.

\subsection*{Schatzmeister}
\begin{tabular}{|l|l|l|}
  \hline
  \textbf{Kandidat} & \textbf{Stimmen} & \textbf{Prozent} \\ \hline
  chrissi\textasciicircum{} (Chris Fiege) & 26 & 100\% \\
  Enthaltung & 0 & 0\% \\
  \hline
\end{tabular}

\elected{Schatzmeister}{chrissi\textasciicircum}{26}{26}
chrissi\textasciicircum{} nimmt die Wahl zum Schatzmeister an.

\subsection*{Beisitzende}
\begin{tabular}{|l|l|l|}
  \hline
  \textbf{Kandidat/in} & \textbf{Stimmen} & \textbf{Prozent} \\ \hline
  comawill (Sebastian Willenborg) & 24 & 92.3\% \\
  Kasalehlia (Hilko Boekhoff) & 22 & 84.6\% \\
  hanhaiwen (Helga Hansen) & 19 & 73.1\% \\
  rohieb (Roland Hieber) & 19 & 73.1\% \\
  Valodim (Vincent Breitmoser) & 19 & 73.1\% \\
  reneger (René Stegmaier) & 15 & 57.6\% \\
  Emantor (Rouven Czerwinski) & 14 & 53.8\% \\
  dadrc (Philipp Specht) & 12 & 46.2\% \\
  DooMMasteR (Steffen Arntz) & 9 & 34.6\% \\
  hellfyre (Matthias Uschok) & 9 & 34.6\% \\
  lichtfeind (Jonas Martin) & 8 & 30.8\% \\
  \hline
\end{tabular}

\elected{Beisitzer}{comawill}{24}{26}
\elected{Beisitzer}{Kasa\-lehlia}{22}{26}
\elected{Beisitzerin}{hanhaiwen}{19}{26}
Kasalehlia und comawill nehmen die Wahl an, rohieb ist schon als stellv.
Vorsitzender gewählt worden und scheidet damit aus. Somit besteht Gleichstand
zwischen Valodim und hanhaiwen. Valodim zieht an dieser Stelle seine Kandidatur
zurück, damit ist hanhaiwen als Beisitzerin gewählt, sie nimmt die Wahl an.

Der neue Vorstand besteht somit aus:
\begin{itemize}
  \item larsan (Lars Andresen), Vorstandsvorsitzender
  \item rohieb (Roland Hieber), stellv. Vorstandsvorsitzender
  \item chrissi\textasciicircum{} (Chris Fiege), Schatzmeister
  \item comawill (Sebastian Willenborg), Beisitzer
  \item Kasalehlia (Hilko Boekhoff), Beisitzer
  \item hanhaiwen (Helga Hansen), Beisitzerin
\end{itemize}

\subsection*{Kassenprüfer}
\consensus{Kassenprüfer nicht neu wählen}
Die Kassenprüfer (shoragan und Juliane) würden das Amt noch ein weiteres Jahr
fortführen. Da die Satzung ihre Amtszeit nicht beschränkt, verzichtet die
Versammlung einstimmig auf die Wahl der Kassenprüfer.


%%%%%%%%%%%
%% TOP 3 %%
%%%%%%%%%%%
\section{Sonstiges}

\subsection{Rücklagenbildung}\label{sec:ruecklagen}
Der Vorstand hatte beschlossen, eine Rücklage von 3000\,€ für die Pufferung von
Einnahmeausfällen zu bilden und möchte sich diesbezüglich das Einverständnis der
Mitgliederversammlung einholen (siehe \ref{sec:finanzbericht-ruecklagen},
Seite~\pageref{sec:finanzbericht-ruecklagen}). Als Beschlusstext wird
vorgeschlagen:

\begin{quote}
  Der Betrag von 3000\,€ wird zurückgelegt, um längerfristige
  Vereinsverbindlichkeiten (Miete, Nebenkosten, laufende Verträge) im Falle
  von Einnahmeausfällen (z.\,B. durch Mitgliederaustritte) abzupuffern.
\end{quote}

Der Betrag von 3000\,€ ist dabei willkürlich gewählt und deckt die bestehenden
Verbindlichkeiten für etwa 4 Monate ab.

\consensus{Rücklage von 3000\,€ für Einnahmeausfälle}
Eine Handabstimmung unter den anwesenden Mitgliedern ergibt eine eindeutige
Mehrheit für diese Rücklage.

\subsection{Stromanbieterwechsel}\label{sec:stromanbieter}
Der Strom im Space 2.0 wird im Moment von der naturstrom AG geliefert. Der
Vorstand hatte sich für einen Ökostromtarif entschieden, nachdem ein Mitglied
sich bereit erklärt hatte, die Differenz zu einem Nicht-Ökostromtarif zu
spenden. Da der Stromverbrauch nach dem Umzug in den Space 2.0 inzwischen aber
über das erwartete Maß hin angestiegen ist und zudem das erwähnte Mitglied nicht
mehr Teil des Vereins ist, wurde angeregt, einen anderen Stromtarif zu wählen.

Mögliche Stromtarife wurden im Vorfeld der Mitgliederversammlung recherchiert
(Gesamtpreise sind auf einen geschätzten Verbrauch von 11{.}000\,kWh/Jahr
gerechnet):

\begin{tabular}{|l|l|p{7cm}|}
  \hline
  \textbf{Tarif} & \textbf{Gesamtpreis} & \textbf{davon Investition in
    erneuerbare Energieerzeugung} \\ \hline
  BS|Energy Gewerbestrom & 2874{,}18\,€/Jahr & 0\,Cent/kWh \\ \hline
  naturstrom (ab 03/2015) & 2982{,}90\,€/Jahr & 1{,}0\,Cent/kWh \\ \hline
  BS|Energy Naturstrom & 2989{,}20\,€/Jahr & 1{,}0\,Cent/kWh \\ \hline
  ews-schoenau & 3025{,}30\,€/Jahr & 0{,}5\,Cent/kWh \\ \hline
  BS|Energy Naturstrom Gold & 3068{,}40\,€/Jahr & 1{,}0\,Cent/kWh \\ \hline
\end{tabular}

Die Differenz zwischen dem billigsten Ökostromtarif (naturstrom ab März 2015)
und BS|Energy Gewerbestrom beträgt 108\,€, die recherchierten
Ökostromtarife haben eine Spannweite von 85\,€.

\consensus{Vorstand soll geeigneten Ökostromtarif wählen}
Zunächst stellt sich die Frage, ob der Verein weiterhin auf einen Ökostromtarif
setzen soll. Ein Meinungsbild zu dieser Frage ergibt eine deutliche Mehrheit für
einen Ökostromtarif. Angesichts der fortgeschrittenen Zeit spricht sich die
Versammlung dafür aus, den Vorstand mit der Wahl eines geeigneten Ökostromtarifs
zu beauftragen.

\subsection{EasterHegg 2015}\label{sec:eh15}
Der Verein hatte sich zusammen mit dem Netz39~e.\,V. aus Magdeburg für die
Ausrichtung der EasterHegg 2015 beworben und den Zuschlag dafür bekommen. Die
EasterHegg wird Ostern 2015 im Jugendzentrum Mühle stattfinden. reneger wirbt
kurz für das Projekt und fordert alle Interessierten auf, bei den wöchentlichen
Besprechungen teilzunehmen.

\subsection{Weltherrschaft}
\postponed
\hyphenation{Stratum}
Der Beschluss über die offizielle Anerkennung des RaumZeitLabors in Mannheim als
Außen\-stelle des Stratum~0~e.\,V.\footnote{siehe
\url{https://events.ccc.de/congress/2014/wiki/index.php?oldid=1713}} wird aus
Zeitgründen auf die nächste Versammlung verschoben.

\begin{description}
  \item[Ende der Versammlung] um 18:20
\end{description}

%%%%%%%%%%%%%%%%%%%%
%% Unterschriften %%
%%%%%%%%%%%%%%%%%%%%
\cleardoublepage
\section*{Unterschriften}
\vspace{0.7cm}
\noindent Protokollführer: \hrulefill\hfill\phantom{c}\par
\vspace{0.7cm}
\noindent Vorstandsvorsitzender: \hrulefill\hfill\phantom{c}\par
\vspace{0.7cm}
\noindent Stellv. Vorsitzender: \hrulefill\hfill\phantom{c}\par
\vspace{0.7cm}
\noindent Schatzmeister: \hrulefill\hfill\phantom{c}\par
\vspace{0.7cm}
\noindent Beisitzer: \hrulefill\hfill\phantom{c}\par
\vspace{0.7cm}
\noindent Beisitzer: \hrulefill\hfill\phantom{c}\par
\vspace{0.7cm}
\noindent Beisitzerin: \hrulefill\hfill\phantom{c}\par
%\vspace{0.7cm}
%\noindent Beisitzer: \hrulefill\hfill\phantom{c}\par

\end{document}

