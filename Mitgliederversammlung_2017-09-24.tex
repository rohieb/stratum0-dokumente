\documentclass{s0minutes}
\usepackage[utf8]{inputenc}
\usepackage[ngerman]{babel}
\usepackage{longtable}
\usepackage{booktabs} % professional tables
\usepackage{multicol}
\usepackage{wasysym}  % for \diameter
\usepackage{textcomp} % for €

\input{vc.tex}

\meetingminutes{\generalassembly}{24. September 2017}{14:00}{Stratum~0,
Hamburger Straße 273 A2, Braunschweig}{27 Mitglieder, keine Gäste}{}{rohieb}

\title{8.\, Mitgliederversammlung}

% Sowas wie "Präsentati-on" braucht doch niemand -.-
\hyphenpenalty=9999

\begin{document}
\maketitle\vspace{-\baselineskip}
\tableofcontents

\enlargethispage{\baselineskip}
\VCfootnote

%%%%%%%%%%%%
%% TOP 0  %%
%%%%%%%%%%%%
\section{Protokoll-Overhead}
\begin{description}
\raggedright
  \item[Eröffnung der Versammlung] durch den Vorstandsvorsitzenden um 14:06
  \item[Wahl der Versammlungsleitung:] Kasalehlia, durch Handzeichen, kein Einspruch
  \item[Wahl der Protokollführung:] rohieb, durch Handzeichen, kein Einspruch
  \item[Quoren:] \quad
    \begin{itemize}[leftmargin=0cm]
      \item zum Tag der Mitgliederversammlung hat der Verein insgesamt 92
        Mitglieder, davon 80 ordentliche Mitglieder und 12 Fördermitglieder
      \item anwesend sind 27 Mitglieder, davon 25 ordentliche, stimmberechtigte
        Mitglieder
      \item 23\% der ordentlichen Mitglieder = \emph{18{,}4 Mitglieder für
        Beschlussfähigkeit}
      \item 50\% der anwesenden, stimmberechtigten Mitglieder = \emph{12{,}5 Mitglieder
        für die Annahme eines Antrags}
    \end{itemize}
  \item[Beschlussfähigkeit:] 25 von geforderten 18{,}4 ordentlichen
    Mitgliedern sind anwesend, die Versammlung ist damit beschlussfähig.
  \item[Notation für Abstimmungen:] (Pro-Stimmen/Contra-Stimmen/Enthaltungen)
\end{description}

%%%%%%%%%%%%%
%% TOP 1   %%
%%%%%%%%%%%%%
\section{Berichte}

%%%%%%%%%%%%%
%% TOP 1.1 %%
%%%%%%%%%%%%%
\subsection{Jahresbericht}

larsan gibt nochmal eine Retrospektive über das gesamte vergangene
Dreivierteljahr. Die Präsentation dazu (mit sehr vielen Bildern) ist auf der
Homepage verfügbar.%
\footnote{\url{https://stratum0.org/wiki/Datei:JahresberichtMV2017.pdf}}

%%%%%%%%%%%%%
%% TOP 1.2 %%
%%%%%%%%%%%%%
\subsection{Finanzbericht}

Emantor gibt als Schatzmeister einen Überblick über die Finanzen im
vergangenen Dreivierteljahr seit der letzten Mitgliederversammlung. Der
vollständige Bericht ist als Präsentation auf der Homepage zu finden%
\footnote{\url{https://stratum0.org/wiki/Datei:Finanzbericht_2017.pdf}}
und wird hier auszugsweise mit den mündlichen Anmerkungen wiedergegeben.

Die Arbeit des Schatzmeisters bestand bisher hauptsächlich daraus, sich in
Workflows einzuarbeiten, und die seltsame Java-Software zur Buchhaltung mit
einer Schicht aus Python drumrum bedienbarer und automatisierbarer zu machen.

\paragraph{Kontenübersicht}

\begin{longtable}{llr}
  \textbf{Nr.} & \textbf{Name} & \textbf{Bestand zum 30.09.2017 [€]}\\
  \midrule
  \endfirsthead
  100         & Barkasse                  & \textcolor{red}{--85{,}40} \\
  \quad 100-1 & Spenden für 3D-Drucker Filament       &      110{,}38 \\
	\quad 100-2 & Pfand für physische Schlüssel         &      360{,}00 \\
	\quad 100-3 & Spenden für Plotter-Material          &      111{,}20 \\
	\quad 100-4 & Spenden für Stick-Material            &      179{,}91 \\
	\midrule
	101         & Erstattungskasse Verbrauchsmaterial   &      148{,}85 \\
	\midrule
	102         & Matekasse                             &      193{,}02 \\
	\midrule
  200024917   & Girokonto                             &  5{.}223{,}32 \\
	\quad 200024917-1 & Rückstellungen Girokonto        &  3{.}160{,}00 \\
	\midrule \midrule
              &  \textbf{Summe}               & \textbf{ 9{.}401{,}28  } \\
	\midrule
\end{longtable}

Der Negativbetrag in der Barkasse entsteht durch eine Mankobuchung, dazu nachher
mehr. Die Unterkonten der Barkasse liegen mit in der Blechbox namens
"`Barkasse"' und balancieren den Fehlbetrag aus, sodass sich noch physisches
Geld in der Box befindet. Da die Spendenboxen aufgelöst wurden (siehe unten),
sollte diese Fall auch nicht mehr auftreten.

\paragraph{Überblick über die Finanzen}
Die Zahlen beziehen sich auf den Zeitraum vom 01.12.2016 bis 30.09.2017.

\begin{longtable}{lr>{\textcolor{red}\bgroup}r<{\egroup}}
  \textbf{Bereich} & \textbf{Einnahmen [€]} & \textbf{Ausgaben [€]} \\
  \midrule
  \endfirsthead
  \multicolumn{3}{c}{\emph{(Fortsetzung von vorheriger Seite)}} \\
  \\
  \textbf{Bereich} & \textbf{Einnahmen [€]} & \textbf{Ausgaben [€]} \\
  \midrule
  \endhead
  \\
  \multicolumn{3}{c}{\emph{(Fortsetzung auf nächster Seite)}} \\
  \endfoot
  \endlastfoot
  Ideeller Bereich: Allgemein       & 17{.}057{,}64 &       739{,}67 \\
  \quad davon Mitgliedsbeiträge     & 11{.}222{,}22 &         0{,}00 \\
  \quad davon Spenden               &  5{.}610{,}42 &         0{,}00 \\
  \quad davon allgemeine Ausgaben   &        0{,}00 &        61{,}40 \\
  \quad davon Kontoführungsgebühren &        0{,}00 &       131{,}22 \\
  \quad davon Pfandgeld Schlüssel   &      120{,}00 &         0{,}00 \\
  \quad davon Vereinsserver         &        0{,}00 &       547{,}05 \\
  \quad davon Bekleidung            &      105{,}00 &         0{,}00 \\
  \midrule
  Ideeller Bereich: Projekte        &  1{.}192{,}63 &   1{.}718{,}41 \\
  \quad davon Bastelmaterial        &        0{,}00 &       624{,}15 \\
  \quad davon Stickmaschine         &        0{,}00 &         0{,}00 \\
  \quad davon Schneidplotter        &      107{,}40 &         0{,}00 \\
  \quad davon 3D-Drucker            &      110{,}38 &        95{,}70 \\
  \quad davon Freifunk              &      195{,}00 &       308{,}71 \\
  \quad davon Schweißen             &      250{,}00 &       210{,}00 \\
  \quad davon CoderDojo             &      529{,}85 &       479{,}85 \\
  \midrule
  Ideeller Bereich: Space           &      229{,}77 &  12{.}368{,}03 \\
  \quad davon Rundfunkgebühren      &      229{,}77 &        17{,}49 \\
  \quad davon Miete und Nebenkosten &        0{,}00 &  11{.}602{,}25 \\
  \quad davon Verbrauchsmaterial    &        0{,}00 &        84{,}34 \\
  \quad davon Einrichtung           &        0{,}00 &       663{,}95 \\
  \midrule
  Vermögensverwaltung               &        0{,}00 &         0{,}00 \\
  \midrule
  Zweckbetriebe                     &        0{,}00 &         0{,}00 \\
  \midrule
  Wirtschaftlicher Geschäftsbetrieb &  5{.}306{,}97 &   6{.}145{,}30 \\
  \quad (Matekasse) && \\
  \midrule
  Mankobuchungen                    &        0{,}00 &       120{,}01 \\
  \midrule\midrule
  Gesamt:                           & 23{.}714{,}74 &  21{.}019{,}14 \\

  \textbf{Gewinn/Verlust Gesamt:} & \textbf{2{.}695{,}60} &          \\
\end{longtable}

Die Einnahmen aus Mitgliedsbeiträgen sind relativ stabil.
Firmenspenden kamen von Pengutronix e.K (1000\,€), UCWare GmbH (1200\,€),
Trilogy GmbH (1000\,€), In-Tech GmbH (1000\,€), EWM AG (Schweißgerät), und 
Festool (verschiedene Werkzeuge/Zubehör).

Aus Projektgeldern wurden ein Laptop für das CoderDojo angeschafft (479\,€),
sowie eine Standbohrmaschine für die Werkstatt (620\,€). Die restlichen Projekte
tragen sich durch Verbrauchsmaterialspenden relativ gut selbst.

Die Zweckbindung der Spendenboxen wurde aufgehoben, sodass das Geld flexibler
nach Bedarf auf die Projekte aufgeteilt werden kann. Dadurch soll verhindert
werden, dass sich zweckgebundene Spenden für eng begrenzte Bereiche ansammeln,
die nicht ausgegeben werden können, weil kein Verbesserungspotenzial vorhanden
ist (bspw.\, bei der Stickmaschine). Die Boxen bleiben aber weiterhin einzeln
bestehen, um die Herkunft der Spenden statistisch nachzuverfolgen.

Die Aufhebung der Zweckbindung wurde kritisiert, da nicht ausreichend darüber
informiert wurde. Daraufhin wurden die Spendenboxen durch Aufkleber "`nicht
zweckgebunden"' markiert, und auf der Diskussionsliste darüber aufgeklärt. Die
bis dahin angefallenen Spenden werden weiterhin als zweckgebunden innerhalb
ihrer Bestimmung behandelt. Ein Betrag von 94{,}08\,€ konnte allerdings nicht mehr
zu einer bestimmten Spendenbox zugeordnet werden.

Die bisher gezahlten Rundfunkgebühren wurden uns zurückerstattet, nachdem wir
in einem Brief an den Beitragsservice dargelegt haben, dass wir keine
Arbeitsplätze eingerichtet haben und somit keine Rundfunkgebühr zahlen müssen.%
\footnote{siehe Blogartikel: \url{https://stratum0.org/blog/posts/2017/06/01/befreifung-gez/}}

In den Mankobuchungen spiegelt sich der Verlust von 120\,€ in der Matekasse
wider, die zwar verbucht wurden, aber auf dem Weg in die Barkasse irgendwo
verloren gegangen sind. Es handelt sich dabei um einen Mitgliedsbeitrag in einem
Briefumschlag.

Es wird kurz über einen Wechsel der Bank diskutiert, um Kontoführungsgebühren
einzusparen. Der Vorstand hatte das bereits auf den Vorstandssitzungen erörtert,
kam aber zu dem Schluss, dass inzwischen alle Banken ähnliche Konditionen haben,
außerdem ist man auf eine Bank mit Bareinzahlung für die Matekasse angewiesen.
Der Migrationsaufwand wird für Mitglieder auch hoch eingeschätzt. Aus Sicht des
Vorstandes überwiegen die Nachteile eines Bankwechsels.

Aktuelle Zahlen zu laufenden Finanzen gibt es jederzeit im dem Stratum~0 Open
Data Portal unter \url{https://data.stratum0.org}.

\paragraph{Gegenüberstellung}
Die Gegenüberstellung der mittleren Einnahmen und laufende Verpflichtungen pro
Monat zeigt im Zeitraum Dezember 2016 bis September 2017 weiterhin einen
deutlichen Überschuss:

\begin{center}
\begin{multicols}{2}
\begin{tabular}{lrr}
  \textbf{Einnahmen} & \multicolumn{2}{c}{\textbf{€/Monat}} \\
  \midrule
  Mitgliedsbeiträge & \diameter & 1{.}247 \\
  Spenden           & \diameter &     633 \\
                    &           &         \\
                    &           &         \\
                    &           &         \\
                    &           &         \\
  \midrule
  \textbf{Gesamt} & \textbf{\diameter} & \textbf{1{.}880} \\
\end{tabular}

\begin{tabular}{l>{\textcolor{red}\bgroup}r<{\egroup}}
  \multicolumn{1}{c}{\textbf{Verpflichtungen}} & \textbf{€/Monat} \\
  \midrule
  Miete, Nebenkosten      & 730 \\
  Strom                   & 285 \\
  Internet                &  42 \\
  Server, Domain          &  57 \\
  Haftpflichtversicherung &  12 \\
  Domain stratum0.org     &   1 \\
  \midrule
  \textbf{Gesamt:} & \textbf{1{.}127} \\
\end{tabular}
\end{multicols}
\end{center}

Im Vergleich zum Jahr 2016 sind die monatlichen Verpflichtungen nur leicht
gestiegen (von 982\,€ auf 1{.}127\,€), während die durchschnittlichen
monatlichen Einnahmen nahezu verdoppelt werden konnten (990\,€ auf 1880\,€ pro
Monat).

\paragraph{Mitgliederentwicklung}
Die Mitgliederzahl steigt weiter, im Moment hat der Verein 92 Mitglieder (79 zur
letzten Mitgliederversammlung im Dezember 2016).

\paragraph{Rückstellungen}
Weiterhin besteht eine Rückstellungen von 3{.}000\,€ zur Abwicklung der
laufenden Geschäfte im Falle von Umsatzeinbrüchen in den Mitgliedsbeiträgen.

Die Rücklage zur Erhöhung der Mietsicherheit von 160\,€ wurde auf der
Vorstandssitzung am 2.\,Dezember 2016 aufgelöst worden, die Rückbuchung wurde
aber bisher noch nicht ausgeführt.

\paragraph{Ausblick}
Dieses Jahr gab es wesentlich mehr Spenden als letztes Jahr, insbesondere durch
viele Firmensponsoren. Für das Jahr 2018 wurden auch schon wieder teilweise
Spenden von Firmen zugesagt.

Die Nebenkostenabrechnung für die Jahre 2014 und 2015 wurden nun bezahlt; und
der monatliche Abschlag wurde um 100\,€ erhöht, sodass weitere
Nebenkostenabrechnungen mit geringerer Nachzahlung ausfallen sollten. Die
Abrechnung für 2016 wird Ende des Jahres erwartet.
Der Stromverbrauch bewegt sich bei ca. 10{.}000\,kWh/a weiterhin stabil auf hohem
Niveau. Außerdem haben die Rechnungsprüfer auch dieses Jahr wieder daran
erinnert, dass zweckgebundene Spenden zeitnah verwendet oder zurückerstattet
werden müssen.

%%%%%%%%%%%%%
%% TOP 1.3 %%
%%%%%%%%%%%%%
\subsection{Bericht der Rechnungsprüfer}

Rechnungsprüferin Angela berichtet von der Kassenprüfung am 2. September
zusammen mit dem zweiten Rechnungsprüfer shoragan, der heute nicht anwesend ist.

Der digitale Teil der Buchhaltung ist nachvollziehbar, der schriftliche Teil
hatte jedoch einige Lücken in der Nachvollziehbarkeit und stimmte somit
stellenweise nicht mit der Digitalversion überein. Es gab auch kein
Übergabeprotokoll zwischen altem und neuem Schatzmeister, was zwar keine Pflicht
ist, aber trotzdem sinnvoll gewesen wäre.

Als größten Mangel sehen sie, dass 120\,€ in der Barkasse fehlen, wie schon vom
Schatzmeister im Finanzbericht erwähnt wurde.

Weiterhin bemängeln sie, wie die Zweckbindung der Spendendosen aufgehoben wurde.
Dieser Beschluss wurde ihrer Meinung nach nicht offen genug kommuniziert, und
die Beschriftung der Spendendosen ("`3D-Drucker"', "`Plotter-Spenden"',
"`Stickmaschine"', etc.) hätte korrigiert werden müssen, um nicht den Eindruck
der Zweckgebundenheit zu erwecken. Beides wurde inzwischen jedoch nachgeholt,
und die eingenommenen Spenden vor diesem Zeitpunkt werden weiterhin
zweckgebunden behandelt.

Aus diesen Gründen können die Rechnungsprüfer die Entlastung des Schatzmeisters
nicht mit gutem Gewissen empfehlen.

Es wurde mit dem Schatzmeister eine Nachprüfung vereinbart, und eine To-Do-Liste
erstellt. Beide Rechnungsprüfer erklären sich auch bereit, das Amt bis zur
nächsten Mitgliederversammlung weiter zu übernehmen.

%%%%%%%%%%%%%
%% TOP 2   %%
%%%%%%%%%%%%%
\section{Entlastung des Vorstands}

Auf die Frage hin, ob jemand möchte, dass die Vorstände einzeln entlastet
werden, meldet sich niemand.

Stattdessen wird die Gegenfrage gestellt, was Entlastung bedeutet. Jemand
erklärt das kurz: Falls der Vorstand entlastet wird, heißt das, dass die
Mitgliederversammlung dem Vorstand gegenüber für alles, was ihr berichtet
wurde, keine Ansprüche mehr geltend machen kann. Die Mitglieder übernehmen
hierbei keine Haftung, sondern sprechen den Vorstand von seiner Haftung in
gewissem Rahmen frei.

Die Versammlungsleitung bittet um Entlastung des gesamten Vorstandes für letzte
Amtsperiode. Es wird per Handzeichen abgestimmt.

\begin{resolution}{MV 2017-09-01}{\vote{\rejected}{9}{12}{1}}{Entlastung des
  gesamten Vorstandes für die letzte Amtsperiode\hfill}{}
  Alle Vorstandsmitglieder enthalten sich.
\end{resolution}

Die Abstimmung hat nicht das geforderte Quorum von 50\% der Stimmen erreicht.
Der Vorstand ist damit vorerst nicht entlastet.

\meetingbreak{von 20 Minuten}
{ \itshape
  Nachträgliche Akkreditierung von 2 ordentlichen Mitgliedern:
  \begin{itemize}[nosep]
    \item Anzahl anwesende ordentliche Mitglieder: \textbf{27}
    \item 50\% der anwesenden ordentlichen Mitglieder für Annahme eines Antrags =
      \textbf{13{,}5}
  \end{itemize}
}

%%%%%%%%%%%%%
%% TOP 3   %%
%%%%%%%%%%%%%
\section{Wahl des Vorstandes}

Als Wahlleitung wird DooMMasteR vorgeschlagen und mit 1 Enthaltung durch
Handzeichen angenommen.

Da diese Mitgliederversammlung nur zur Verlängerung der Amtsperiode des
Vorstandes bis zum Januar dienen soll, wird die Bestätigung des gesamten
Vorstandes bis zur nächsten Mitgliederversammlung vorgeschlagen.

Zuerst wird aber über den Termin der nächsten Mitgliederversammlung abgestimmt.
Es wird der Sonntag, der 14.\ Januar 2018 vorgeschlagen und per Handzeichen
abgestimmt.

\begin{resolution}{MV 2017-09-03}{\vote{\adopted}{26}{0}{1}}{Nächste
  Mitgliederversammlung am Sonntag, 14.\ Januar 2018}{}
\end{resolution}

Die Vorstandsmitglieder stellen sich noch einmal kurz vor.

Die Wahlleitung fragt, ob eine Abstimmung über einzelne Vorstandsmitglieder
gefordert wird. Niemand fordert das. Es wird vorgeschlagen, die Bestätigung des
Vorstands als Blockwahl per Handzeichen vorzunehmen. Alle sind damit
einverstanden.

\begin{resolution}{MV 2017-09-02}{\vote{\adopted}{27}{0}{0}}{Bestätigung des
  bestehenden Vorstands bis zur nächsten Mitgliederversammlung}{}
  Damit sind im Amt bestätigt:
  \begin{description}
    \item[Vorstandsvorsitzender:] Kasalehlia (Hilko Boekhoff), nimmt die Wahl an
    \item[stellv. Vorstandsvorsitzender:] rohieb (Roland Hieber), nimmt die Wahl
      an
    \item[Schatzmeister:] Emantor (Rouven Czerwinski), nimmt die Wahl an
    \item[Beisitzer:] chrissi\^{} (Chris Fiege), larsan (Lars Andresen) und
      reneger (René Stegmaier), alle nehmen die Wahl an
  \end{description}
\end{resolution}

Die gewählten Rechnungsprüfer, shoragan und Angela, würden beide das Amt
weiterhin übernehmen. Niemand fordert, die Rechnungsprüfer neu zu wählen.
Die Rechnungsprüfer bleiben also weiterhin im Amt.

Die Wahlleitung übergibt zurück an die Versammlungsleitung.

%%%%%%%%%%%
%% TOP 4 %%
%%%%%%%%%%%
\section{Sonstiges}

Folgende Punkte werden aufgrund der fortgeschrittenen Zeit vertagt:
\begin{itemize}
  \item Weltherrschaft
  \item Keysigning-Party
  \item Anerkennung des RaumZeitLabors als Außenstelle des Stratum~0
\end{itemize}

\meetingend{15:48}
\end{document}

% vim: set et ts=2 sw=2 sts=2 tw=80:
