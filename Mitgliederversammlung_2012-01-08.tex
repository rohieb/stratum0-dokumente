% vim:tw=80 ts=2 et sw=2 indentexpr= :
\documentclass[a4paper,12pt]{scrartcl}
\usepackage[utf8]{inputenc}
\usepackage[T1]{fontenc}
\usepackage[ngerman]{babel}
\usepackage{libertine} % kann man notfalls auch ignorieren, wenns nicht da ist
\usepackage{textcomp} % notfalls für €
\usepackage{stratum0doc}
\usepackage[colorlinks=false]{hyperref}
\usepackage{graphicx}
\usepackage{savefnmark}

\addtolength{\textwidth}{-10pt}
\addtolength{\marginparwidth}{10pt}

\title{1.~Mitgliederversammlung des Stratum~0~e.~V.}
\date{8.~Januar~2012}

\begin{document}
\maketitle

%%%%%%%%%%%
%% TOP 0 %%
%%%%%%%%%%%
\section{Eröffnung}
\begin{description}
  \item[Zeit:] 9. Januar 2012, 15:00
  \item[Ort:] beyti Grillhaus, Bohlweg; später Plaza des Informatikzentrums,
    Mühlenpfordtstraße
  \item[Anwesend:] 28 Mitglieder, davon 27 mit Stimmrecht
  \item[Wahl des Versammlungsleiters:] Valodim durch Handzeichen; nimmt die
    Wahl an
  \item[Protokoll:] rohieb erklärt sich bereit, keine Gegenstimmen
  \item[Veranstaltung eröffnet] durch den Versammlungsleiter um 15:50
\end{description}

Der Versammlungsleiter stellt fest, dass die Mitgliederversammlung satzungsgemäß
einberufen wurde und beschlussfähig ist. Die Einladung wurde am 23.  Dezember
2011 an die letzte bekannte Adresse jedes Mitglieds versandt. Von insgesamt 32
Mitgliedern sind 27 stimmberechtigte Mitglieder anwesend, dies stellt mehr als
23\%, die in der Satzung geforderte Mindestanzahl an Mitgliedern, dar.

Die Tagesordnung wird ohne Enthaltungen und ohne Gegenstimmen angenommen.

%%%%%%%%%%%
%% TOP 1 %%
%%%%%%%%%%%
\section{Berichte}
\subsection{Bericht des Vorstandes}
Valodim stellt zuerst den neu hinzugekommenen Mitglieder die amtierenden
Vorstandsmitglieder vor. Darauf fasst er die Aktivitäten des Vorstands im
vergangenen Jahr zusammen:
\begin{description}
  \item[30. August 2010:] 1. Vorstandssitzung.\footnote{Vollständiges Protokoll:
    \url{https://stratum0.org/wiki/Vorstandssitzung\_2011-08-30}}
    Themen: Überblick über Immobiliensituation, Besichtigungen, Eintragung,
    Konto.
  \item[24. September 2012:] positive Rückmeldung vom Amtsgericht über die
    Eintragung ins Vereinsregister
  \item[4. Oktober 2010:] Eröffnung eines Vereinskontos bei der
    Braunschweigischen Landessparkasse, verlief nicht ganz reibungslos aufgrund
    der laufenden EDV-Umstellung der Bank
  \item[30. November 2010:] Erster Kontoauszug
  \item[11. Dezember 2010:] 2. Vorstandssitzung.\footnote{Vollständiges
    Protokoll: \url{https://stratum0.org/wiki/Vorstandssitzung\_2011-12-11}}
    Themen: Finanzen, Bericht Raum-Situation, Mitgliederversammlung,
    Weihnachtsfeier
  \item[27.-20. Dezember 2010:] Streaming der Vorträge vom 28. Chaos
    Communication Congress im Rahmen des Programms "`No Nerd Left Behind"', in
    Zusammenarbeit mit der Fachgruppe Informatik der TU Braunschweig. Es waren
    auch viele Nicht-Studenten in den Räumen der TU anzutreffen.
\end{description}

Die Suche nach Räumlichkeiten lief nebenher und stagnierte teilweise, da der
Immobilienmarkt kaum veränderlich war. Darum wurden auf der 2. Vorstandssitzung
wöchentliche Treffen zur Beobachtung des Immobilienmarktes angeregt, von denen
das erste Treffen am 18. Dezember auch gleich zur Besichtigung der Räumlichkeit
an der Hamburger Straße am 20. Dezember bzw. in wiederholter Form am 3. Januar
führte (siehe \ref{top:space}).

Gleichzeitig waren während des Jahres im Rahmen der Aktion "`Zeitabgleich"'
mehrere Mitglieder des Vereins in verschiedenen Hackerspaces in Deutschland zu
Gast und haben Werbung für unseren Verein gemacht: Neo Bechstein, ktrask und
m00lean Anfang September beim \emph{chaosdorf} in Düsseldorf zum Kongress
OpenRheinRuhr, whisp in der \emph{c-base} in Berlin, mehrere Mitglieder in
Hannover bei der \emph{leitstelle511} Anfang August, und blinry live auf dem
28. Chaos Communication Congress in Berlin Ende Dezember.

Außerdem wurde auf der 1. Vorstandssitzung der Beschluss getroffen, zuerst keine
Gemeinnützigkeit beim Finanzamt zu beantragen, um die Rückabwicklung der
Mitgliedsbeiträge zum Wohle der Mitglieder im Falle eines Scheiterns des
Vereins einfacher zu gestalten. Andernfalls könnten die eingenommen Beiträge nur
einem weiteren, gemeinnützigen Verein zu Gute kommen. Grundsätzlich wird der
Status der Gemeinnützigkeit aber weiterhin angestrebt, sobald der Verein stabil
läuft und eine Räumlichkeit gefunden ist. Aus Valodims Sicht wird aber dieser
Grund gegen die Gemeinnützigkeit aktuell im Laufe der Zeit immer weiter
hinfällig. Außerdem ist die Gemeinnützigkeit zwar in der Satzung festgehalten,
aber sie interessiert kaum jemanden, solange der Verein keine
Steuervergünstigungen beim Finanzamt beantragt.

\section{Bericht des Schatzmeisters und der Kassenprüfer}
Wie schon erwähnt hat sich die Einrichtung des Kontos etwas verzögert, wodurch
erste Überweisungen von Mitgliedern nicht ausgeführt werden konnten. Dies ist
aber inzwischen nach einem nochmaligen Gespräch des Vorstandes mit der Bank
behoben. Außerdem hat sich gezeigt, dass uns pro eingehender Überweisung 5 Cent
Kontoführungsgebühr angerechnet werden, worüber der Vorstand bei der Eröffnung
des Kontos nicht aufgeklärt wurde. Der Schatzmeister wird diese Situation noch
zeitnah mit der Bank besprechen; es besteht wohl diesbezüglich
Verhandlungsbereitschaft seitens der Bank.

An Mitgliedsbeiträgen sind seit Gründung des Vereins im Juni 2010 bis zum
Zeitpunkt der letzten Kontenübersicht einige Tag vor der Versammlung bisher
insgesamt 2.088€ überwiesen worden. Der Schatzmeister weist aber auch scharf
darauf hin, dass zu diesem Zeitpunkt noch 1.490€ an offenen Beiträgen
ausstanden. Davon wurde wiederum vor Beginn der Versammlung ein Großteil durch
Barzahlung ausgeglichen, sodass nun noch etwa 200€ an Forderungen von
Mitgliedern offen sind. Außerdem sind dem Verein 252{,}79€ an Spendengelder
zugekommen, weiterhin wurden mit LINET Services und BEL|NET zwei Firmensponsoren
gewonnen.

Auf der anderen Seite sind als Ausgaben aufgelaufen:
\begin{itemize}
  \item 26{,}78€ Notarkosten für die Eintragung,
  \item 8{,}90€ Kontoführungsgebühr
  \item außerdem noch etwa 70€ Bearbeitungskosten des Amtsgerichts für die
    Eintragung ins Vereinsregister, die bisher von Valodim ausgelegt wurden, und
    die er noch nicht vom Verein zurück erhalten hat.
\end{itemize}

Die Kassenprüfer konnten alle Kontenbewegungen nachvollziehen, bis auf den
genauen Betrag der Spenden, der sich in ihrer Rechnung minimal unterscheidet,
was wohl auf einen Übertragungsfehler seitens des Schatzmeisters zurückzuführen
ist. Das verbuchte Gesamtguthaben am Beginn der Veranstaltung beläuft sich
demnach auf 3{.}320€.

Die Kassenprüfer merken noch an, dass die Aufzeichnungen des Schatzmeisters
etwas schwer nachzuvollziehen sind. Sie erklären sich aber bereit, zusammen
mit dem Schatzmeister eine nachvollziehbarere Methode zu finden.

\subsection{Entlastung des Vorstandes und des Schatzmeisters}
Terminar spricht hier den Punkt an, dass zwischendurch teilweise zu viel Stille
herrschte und der Vorstand nur schwer zu erreichen war. Er sieht es nicht als
grundsätzliches Problem an, wenn zwischendurch Funkstille herrscht, und es
nichts zu berichten gibt, nur sollten voraussehbarere Abwesenheitszeiten
angekündigt und eine Vertretung bestellt werden, sodass zumindest zu jedem
Zeitpunkt eine minimale Erreichbarkeit besteht. Anscheinend gab es zum selben
Zeitpunkt auch noch technische Probleme auf der öffentlichen Mailingliste,
m00lean wirft aber ein, dass diese schnell behoben wurden. Grundsätzlich
schließen sich aber Valodim und Neo Bechstein den Ausführungen von Terminar
an und sehen Optimierungsbedarf in solchen Fällen.

\vote{Entlastung des Vorstands}{18}{0}{9}
Für die Entlastung des Vorstandes wird per Handzeichen abgestimmt. 18 Entitäten
stimmen dafür, 9 enthalten sich, es gibt keine Gegenstimmen. Der Vorstand wird
entlastet.

\emph{[16:26 bis 17:05: Pause, Relokation auf die Plaza des Informatikzentrums]}

%%%%%%%%%%%
%% TOP 2 %%
%%%%%%%%%%%
\section{Miete einer Räumlichkeit}
\label{top:space}
Es steht zur Diskussion, eine Räumlichkeit in der Hamburger Straße 273a
(Schimmel-Hof) zu mieten. Dem Vorstand liegt ein Expose der Firma nowo vor,
welches herumgereicht wird. Zuvor hatte schon eine Besichtigung der Räumlichkeit
mit mehreren Mitgliedern stattgefunden, Bilder davon sind im Wiki zu finden.%
\footnote{siehe
\url{https://stratum0.org/wiki/Kategorie:Spacebesichtigung_2012-01-03}}

Eckdaten:
\begin{description}
  \item[Größe:] 60~m$^2$
  \item[Kaltmiete:] 300€ (5€/m$^2$)
  \item[Geschätzte Warmmiete:] etwa 500€
  \item[Kaution:] etwa 1500€
  \item[Räume:] Flur, Küche, Bad mit Dusche, zwei weitere Räume\footnote{%
    Grundriss siehe \url{https://stratum0.org/wiki/Datei:Space-Grundriss.svg}},
    bezugsfertig renoviert
\end{description}

Die Kaution stellt kein Problem dar, Schatzmeister und Kassenprüfer sind sich in
diesem Punkt einig. Ein großer Vorteil ist die Nähe zur TU Braunschweig und die
Lage in einem Gewerbegebiet, was auch spätnachts noch lautstärkere Aktionen
erlaubt (in direkter Nachbarschaft ist außerdem ein Tanzsportverein). Des
weiteren ist im selben Haus eine Softwarefirma angesiedelt, zu der man Kontakte
knüpfen könnte.

Der Vermieter scheint uns wohl gesonnen zu sein und hatte schon im Vorfeld das
Objekt für uns reserviert und sein Angebot aus dem Internet genommen. Außerdem
ist die Miete allein durch die Mitgliederbeiträge abgedeckt, sodass keine
Abhängigkeit von Sponsoren besteht.

Der einzige Nachteil scheint die Größe zu sein, etwas mehr Platz wird von vielen
Mitgliedern gewünscht. Es wird jedoch angemerkt, dass durchaus die Möglichkeit
eines Umzugs besteht, falls sich in Zukunft bessere Möglichkeiten bieten.
Schließlich ist eine kleine Räumlichkeit für den Anfang besser als garkeine.

Es wird nach weiteren Objekten gefragt, die in der letzten Zeit besichtigt oder
beobachtet wurden.
\begin{itemize}
  \item m00lean hatte eine ehemalige Kneipe gefunden, die zur Vermietung stand.
    Auf seine mehrmalige Anfrage hatte sich allerdings niemand gemeldet.
  \item reneger hatte über seinen Arbeitgeber die Möglichkeit eines Angebots
    eingeräumt bekommen, dort liegen aber noch kein konkretes Angebot vor.
    Außerdem wäre dieser von seinem Arbeitgeber gesponsert.
  \item Der gemeinsam besichtigte Keller in der Kupfertwete/Beckenwerkerstraße
    schied wegen Feuchtigkeit und Renovierungsbedürfnis aus.
\end{itemize}

\vote{Miete der Räumlichkeit Schimmel-Hof durch den Vorstand}{25}{0}{2}
Es wird per Handzeichen darüber abgestimmt, ob dem Vorstand die Erlaubnis
eingeräumt werden soll, das Objekts im Schimmel-Hof zu mieten. 25 Mitglieder
stimmen dafür, 0 Mitglieder stimmen dagegen, es gibt 2 Enthaltungen. Dem
Vorstand wird die Erlaubnis gegeben, das Objekt zu mieten. Außerdem wird
abgesprochen, dass der Mietvertrag vorher auf der Mailingliste zur Diskussion
gestellt wird.

\subsection{Rauchverbot}
\vote{Rauchverbot im Hackerspace}{24}{3}{0}
Es wird per Handzeichen darüber abgestimmt, ob ein vollständiges Rauchverbot in
den zukünftigen Räumlichkeiten gelten soll. Für ein Rauchverbot stimmen 24
Mitglieder, 3 Mitglieder stimmen dagegen, es gibt keine Enthaltungen. Es gibt
ein vollständiges Rauchverbot in den neuen Räumlichkeiten.

\subsection{Vermietung an Externe}
\vote{Vermietung an Externe}{1}{20}{6}
Es wird darüber abgestimmt, ob die Räumlichkeiten an Externe vermietet werden
dürfen, die nicht Mitglieder des Vereins sind. Neo Bechstein merkt an, dass
wahrscheinlich sehr viele technische Geräte in den Räumlichkeiten vorhanden sein
werden, die bei Partys zu Schaden kommen können. Dagegen ist von Mitgliedern zu
erwarten, dass sie dieser Problematik gegenüber ausreichend sensibilisiert sind.
Die Abstimmung erfolgt per Handzeichen, es stimmt 1 Mitglied für die Vermietung
an Externe, 20 Mitglieder stimmen dagegen, es gibt 6 Enthaltungen. Die neuen
Räumlichkeiten werden nicht an Externe vermietet.

\subsection{Vergabe von Schlüsseln}
Es eröffent sich die Problematik der Vergabe von Schlüsseln zur zukünftigen
Räumlichkeit an Mitglieder. Nach heftiger Diskussion wird der Konsens gefasst,
dass die Entscheidung über zu vergebende Schlüssel an den Vorstand übertragen
wird, mit der Aussicht, dass der einzige Schlüssel für die Haustür beim 1.
Vorsitzenden verbleibt und zunächst gegen Pfand 6 Schlüssel für die obere Tür an
Mitglieder herausgegeben werden. Im Wiki\footnote{siehe %
\url{https://stratum0.org/wiki/Schl\%C3\%BCssel}} soll festgehalten werden,
welche Mitglieder zur Zeit Schlüssel besitzen. Zudem ist abzusehen, dass zeitnah
ein elektronisches Schließsystem eingebaut wird, wodurch weitere Schlüssel unter
Umständen ohne großen Aufwand verteilt werden können.

\emph{[Ein Mitglied verlässt die Versammlung, es verbleiben 26 stimmberechtige
Mitglieder]}

\vote{Regelung der Schlüsselvergabe durch Vorstand}{25}{0}{1}
Die Abstimmung erfolgt per Handzeichen. Für die Regelung der Schlüsselvergabe
durch den Vorstand sprechen sich 25 Mitglieder aus, es gibt keine Gegenstimmen
und 1 Enthaltung. Der Vorstand übernimmt die Aufgabe der Schlüsselverteilung an
Mitglieder.

\emph{[Der 1. Vorsitzende lässt zu Protokoll geben, dass jh und Silberwoelfin
ihn im Chat diskreditieren.]}

\emph{[Es gibt eine Pause von 18:09 bis 18:21]}

\subsection{Hausordnung}
\postponed
Es besteht Konsens, dass nach der Anmietung der Räumlichkeit eine Hausordnung
durch ein entsprechendes Gremium erarbeitet werden soll. Kinder sollen auf
eigene Verantwortung mitgebracht werden dürfen. Die Entscheidung über die
Hausordnung wird auf einen späteren Zeitpunkt vertagt.

%%%%%%%%%%%
%% TOP 3 %%
%%%%%%%%%%%
\section{Vorstandswahlen}
Die Amtszeit des auf der Gründungsversammlung gewählten Vorstandes\footnote{%
Vorstandsvorsitzender: Vincent Breitmoser, stellv.\ Vorsitzender: Michael Klug,
Schatzmeister: Julien Deseke (ehem. Jassmann), Beisitzer: Roland Hieber, Lars
Andresen, Ortwin Regel} läuft mit der ersten Mitgliederversammlung
ab.\footnote{Protokoll siehe
\url{https://stratum0.org/wiki/Datei:Gr\%C3\%BCndungsversammlung_2011-07-23.pdf}}

Als Wahlleiter wird durch Handzeichen mit absoluter Mehrheit Johannes Starosta
bestimmt.

\subsection{Vorstandsvorsitzender}
Gewählt wird durch Handzeichen. Jedes Mitglied hat eine Stimme.
\elected{Vorstands\-vorsitzender}{Vincent Breitmoser}{24}{26}
\begin{itemize}
  \item Kandidat 1: Vincent Breitmoser (24 Stimmen)
  \item Enthaltungen: 2 Stimmen
\end{itemize}
Vincent Breitmoser nimmt die Wahl an.

\subsection{Stellv. Vorsitzender}
Gewählt wird durch Handzeichen. Jedes Mitglied hat eine Stimme.
\elected{stellv. Vorsitzender}{Michael Klug}{24}{26}
\begin{itemize}
  \item Kandidat 1: Michael Klug (24 Stimmen)
  \item Enthaltungen: 2 Stimmen
\end{itemize}
Michael Klug nimmt die Wahl an.

\newpage

\subsection{Schatzmeister}
Gewählt wird durch Handzeichen. Jedes Mitglied hat eine Stimme.
\elected{Schatzmeister}{Julien Deseke}{24}{26}
\begin{itemize}
  \item Kandidat 1: Julien Deseke (ehem. Jassmann) (24 Stimmen)
  \item Enthaltungen: 2 Stimmen
\end{itemize}
Julien Deseke (ehem. Jassmann) nimmt die Wahl an.

\emph{[Ein Mitglied verlässt die Versammlung, es verbleiben 25 stimmberechtige
Mitglieder]}

\subsection{Beisitzer}
Gewählt wird durch geheime Wahl. Jedes Mitglied hat beliebig viele Stimmen und
darf jedem Kandidaten höchstens eine Stimme geben. Die drei Kandidaten mit den
meisten Stimmen werden gewählt, vorausgesetzt, die Anzahl der Stimmen macht als
die Hälfte der stimmberechtigen Mitglieder aus. Falls es zu Uneindeutigkeiten
kommt, gibt es Stichwahlen.

\elected{Beisitzer}{Roland Hieber}{19}{25}
\elected{Beisitzer}{Lars Andresen}{18}{25}
\elected{Beisitzer}{Ortwin Regel}{17}{25}
\begin{itemize}
  \item Beginn des Wahlgangs um 18:38, es werden 25 Stimmzettel ausgegeben.
  \item Ende des Wahlgangs um 18:42, es wurden 25 Stimmzettel zurückerhalten.
  \item Kandidat 1: Ortwin Regel (17 Stimmen)
  \item Kandidat 2: Lars Andresen (18 Stimmen)
  \item Kandidat 3: Nico Grasshoff (13 Stimmen)
  \item Kandidat 4: Rebecca Husemann (16 Stimmen)
  \item Kandidat 5: Roland Hieber (19 Stimmen)
  \item Kandidat 6: Mathias Erdmann (4 Stimmen)
\end{itemize}
Roland Hieber, Lars Andresen und Ortwin Regel nehmen die Wahl an.


%%%%%%%%%%%
%% TOP 4 %%
%%%%%%%%%%%
\section{Logo}
Im Vorfeld der Versammlung wurden mögliche Logo-Entwürfe eingereicht, welche in
einer formlosen Abstimmung unter den Mitgliedern aussortiert wurden. Die vier
Favoriten dieser Abstimmung werden herumgereicht.

\begin{figure}[p]
  \centering
  \includegraphics[width=0.5\textwidth]{Sanduhr.png}
  \caption{Vorschlag 1: "`Sanduhr"'}
\end{figure}
\vspace{20pt}
\begin{figure}[p]
  \centering
  \includegraphics[width=0.5\textwidth]{Sanduhr_fractal.png}
  \caption{Vorschlag 2: "`Sanduhr, rekursive Version"'}
\end{figure}
\vspace{20pt}
\begin{figure}[p]
  \centering
  \includegraphics[width=0.5\textwidth]{Zifferblatt.png}
  \caption{Vorschlag 3: "`Ziffernblatt"'}
\end{figure}
\vspace{20pt}
\begin{figure}[p]
  \centering
  \includegraphics[width=0.5\textwidth]{Stratumscience.png}
  \caption{Vorschlag 4: "`Stratum~0 Science"'}
\end{figure}

Nachdem alle Mitglieder von den Vorschlägen Notiz genommen haben, wird per
Handzeichen über das neue Vereinslogo abgestimmt. Jedes Mitglied hat beliebig
viele Stimmen und darf jedem Vorschlag höchstens eine Stimme geben. Der
Vorschlag mit den meisten Stimmen wird gewählt, bei Gleichstand gibt es eine
Stichwahl.

\begin{itemize}
  \item Vorschlag 1: "`Sanduhr"' (15 Stimmen)
  \item Vorschlag 2: "`Sanduhr, rekursive Version"' (5 Stimmen)
  \item Vorschlag 3: "`Ziffernblatt"' (15 Stimmen)
  \item Vorschlag 4: "`Stratum~0 Science"' (7 Stimmen)
\end{itemize}

Es gibt eine Stichwahl zwischen Vorschlag 1 und Vorschlag 3. Es wird per
Handzeichen abgestimmt, jedes Mitglied hat eine Stimme.
\elected{neues Logo}{Vorschlag "`Sanduhr"'}{11}{25}
\begin{itemize}
  \item Vorschlag 1: "`Sanduhr"' (11 Stimmen)
  \item Vorschlag 3: "`Ziffernblatt"' (10 Stimmen)
  \item Enthaltung: 4 Stimmen
\end{itemize}

Der Vorschlag 1 "`Sanduhr"' wird als neues Vereinslogo angenommen.\footnote{%
siehe \url{https://stratum0.org/wiki/Logo}}


%%%%%%%%%%%
%% TOP 4 %%
%%%%%%%%%%%
\section{Änderung der Beitragsordnung}
\subsection{Lastschriftverfahren nicht mehr zulassen}
\vote{Lastschrift\-verfahren nicht mehr zulassen}{24}{0}{1}
Die Beitragsordnung soll wie folgt geändert werden. §3~Abs.~1 mit dem aktuellen
Wortlaut
\begin{quote}
  Die Zahlung des Beitrages erfolgt im Lastschriftverfahren oder per
  Überweisung.
\end{quote}
soll durch den neuen Absatz ersetzt werden:
\begin{quote}
  Die Zahlung des Beitrages erfolgt per Überweisung.
\end{quote}

Hintergrund des Antrages ist, dass das Lastschriftverfahren bei unserem
aktuellen Banktarif Kosten verursacht. Die Barzahlung soll wie bisher bestehen
bleiben.

Es wird per Handzeichen abgestimmt. Für den Antrag stimmen 24 Mitglieder, es
gibt keine Gegenstimmen und 1 Enthaltung. Die Beitragsordnung wird wie
beschrieben geändert.\footnote{Aktuelle Version: siehe
\url{https://stratum0.org/wiki/Beitragsordnung}}\saveFN\FNUrlBeitrO

\subsection{Neuregelung ermäßigter Beitragssatz}
\vote{Neuregelung ermäßigter Beitragssatz}{25}{0}{0}
Die Beitragsordnung soll wie folgt geändert werden. §1~Abs.~2 mit dem aktuellen
Wortlaut
\begin{quote}
  Schüler, Studenten und Hartz-IV-Empfänger zahlen nach Einreichung eines
  entsprechen den Nachweises beim Vorstand den ermäßigten Beitrag von 12€ pro
  Monat.
\end{quote}
soll durch die folgende Formulierung ersetzt werden:
\begin{quote}
  Schüler, Studenten, Empfänger von Sozialgeld oder Arbeitslosengeld II
  einschließlich Leistungen nach § 22 ohne Zuschläge oder nach § 24 des Zweiten
  Buchs des Sozialgesetzbuchs (SGB II), Empfänger von Ausbildungsförderung nach
  dem Bundesausbildungsförderungsgesetz (BAföG) haben die Möglichkeit, einen
  ermäßigten Beitrag von 12€ pro Monat zu zahlen. Ein entsprechender Nachweis
  muss dem Vorstand auf Verlangen zugänglich gemacht werden.
\end{quote}

Es ist zu beachten, dass die neue Regelung nicht mehr in allen Fällen einen
Nachweis über die Berechtigung erfordert, sondern nur noch in Fällen, in denen
Unklarheit besteht oder in denen Mitglieder zu wenig zahlen. Insofern erfordert
dies auch weniger Arbeit für den Schatzmeister.

\vote{Neuregelung ermäßigter Beitragssatz}{25}{0}{0}
Die Abstimmung erfolgt per Handzeichen. 25 Mitglieder stimmen für den Antrag, es
gibt keine Enthaltungen und keine Gegenstimmen. Die Beitragsordnung wird wie
beschrieben geändert.\useFN\FNUrlBeitrO


%%%%%%%%%%%
%% TOP 4 %%
%%%%%%%%%%%
\section{Sonstiges}
\subsection{Zusammenarbeit mit PiratenPC}
\postponed
Neo Bechstein regt eine Zusammenarbeit mit dem Verein PiratenPC an, der alte
Computer aufarbeitet und an bedürftige Haushalte abgibt. Die Zusammenarbeit
könnte sich zum Beispiel auf gemeinsames Basteln an PCs erstrecken. Der
Tagesordnungspunkt wird vertagt, da aufgrund der noch nicht vorhandenen
Räumlichkeit im Moment noch nicht abzusehen ist, ob eine Zusammenarbeit
stattfinden kann.

\subsection{Geschlossene Mailingliste/Mailverteiler für Mitglieder}
\postponed
Es wird gewünscht, eine Kontaktmöglichkeit für die Mitglieder untereinander
einzurichten, die auch nur auf Mitglieder beschränkt ist. Gerade finanzielle
Dinge sollten nicht über die öffentliche Mailingliste verteilt werden, außerdem
wäre eine solche Kontaktmöglichkeit für Einladungen zu Mitgliederversammlungen
o.~ä. nützlich, da aufgrund des hohen Informationsvolumens auch nicht alle
Mitglieder auf dem öffentlichen Mailverteiler mitlesen. Da erwartet wird, dass
sich die Meinungen hierzu nach der Miete einer Räumlichkeit noch ändern, und
angesichts der fortgeschrittenen Zeit wird das Thema jedoch nicht weiter
diskutiert. Ein Meinungsbild unter den Mitgliedern ergibt aber, dass Bedarf nach
einer zuverlässigen und weniger frequentierten Möglichkeit der Benachrichtigung
besteht.

\subsection{Spenden}
\postponed
Spenden haben zur Zeit einen eher geringeren Stellenwert, daher wird das Thema
verschoben, bis eine Räumlichkeit bezogen ist. Im Moment gibt es die
Möglichkeit, per Flattr-Button\footnote{Flattr-Button:
\url{https://flattr.com/profile/stratum0}} oder per Banküberweisung%
\footnote{Kontodaten siehe \url{https://stratum0.org/wiki/Spenden}} zu spenden.
Der Einrichtung eines PayPal-Kontos stehen die Mitglieder gespalten gegenüber.

\subsection{Mitgliederwerbung, Soziale Netze}
\postponed
blinry und Neo Bechstein wollen Werbung für den Hackerspace machen, blinry hat
auch schon Flyer entworfen. joke würde sich außerdem um die Bestellung von
T-Shirts kümmern. Zusätzlich stellt sich die Frage, wie und ob sich der Verein
zu Werbezwecken in weiteren sozialen Netzen (neben dem schon existierenden
Account auf Twitter\footnote{siehe \url{https://twitter.com/stratum0}})
darstellen will. Valodim schlägt vor, dies nach der Miete einer Räumlichkeit in
ein Gremium auszulagern. Es gibt keine gegenteiligen Meinungen dazu, jedoch
würde Terminar gerne auf der nächsten Mitgliederversammlung darüber reden.

\subsection{Verpixelung auf öffentlichen Bildern}
\novote
Einige Mitglieder hatten Unmut gegenüber öffentlich sichtbaren Bildern von
ihnen geäußert. Als Lösungsmöglichkeiten wird vorgeschlagen, einen internen
Mitgliederbereich aufzubauen, und auf öffentlichen Fotos grundsätzlich alle
erkennbaren Gesichter zu verpixeln. Falls Mitglieder nicht verpixelt werden
wollen, können sie dies ankündigen. Es wird aber auch betont, dass (vor der
Fotografie) jederzeit gefragt werden kann, ob die abgebildeten Personen
einverstanden sind.

Der Vorschlag scheint auf Zustimmung zu stoßen, es wird jedoch kein Beschluss in
dieser Sache gefasst.

\emph{[Ein Mitglied verlässt den Raum]}

\subsection{Internetzugang, Hosting-Dienste}
\consensus{Internet vorerst als DSL, bei Bedarf Gremium bilden}
Zuletzt wird darüber diskutiert, welche Dienste in den anzumietenden
Räumlichkeiten gehostet werden sollen und wie die Internetanbindung zustande
kommt. Grundsätzlich stellt sich aber die Frage, ob der Verein einen Hackerspace
oder ein Rechenzentrum in den Räumlichkeiten betreiben will. Die Diskussion
führt schnell dazu, dass vorerst eine handelsübliche DSL-Verbindung mit
hoher Geschwindigkeit ausreichen sollte, alles andere kann später entschieden
werden. Zusätzlich wird darauf hingeweisen, dass schnellere, aber
trafficbasierte Internetangebote (z.~B. von Gärtner Datensysteme im selben Haus
im Schimmel-Hof) zu unkontrollierbaren Kosten führen könnten.

Es wird als Konsens festgehalten, dass der Vorstand vorerst einen
handelsüblichen DSL-Anschluss bestellen wird, und alles andere zu diesem Thema
dann bei Bedarf in ein Gremium ausgelagert werden soll.

\begin{description}
\item[Veranstaltung geschlossen] durch den Versammlungsleiter um 20:02.
\end{description}

\setlength{\textwidth}{0.5\textwidth}
\vspace{0.7cm}
\noindent Protokollführer: \hrulefill

\vspace{0.7cm}
\noindent Vorstandsvorsitzender: \hrulefill

\vspace{0.7cm}
\noindent Stellv. Vorsitzender: \hrulefill

\vspace{0.7cm}
\noindent Schatzmeister: \hrulefill

\vspace{0.7cm}
\noindent Beisitzer: \hrulefill

\vspace{0.7cm}
\noindent Beisitzer: \hrulefill

\vspace{0.7cm}
\noindent Beisitzer: \hrulefill

\end{document}
% vim: set tw=80 et sw=2 ts=2:
