\documentclass{s0minutes}
\usepackage[utf8]{inputenc}
\usepackage[ngerman]{babel}
\usepackage{longtable}
\usepackage{booktabs} % professional tables
\usepackage{multicol}
\usepackage{wasysym}  % for \diameter
\usepackage{textcomp} % for €

\meetingminutes{\generalassembly}{4. Dezember 2016}{14:00}{Stratum 0,
Braunschweig}{27 stimmberechtigte Mitglieder,\\ & 1 nicht stimmberechtigtes
Mitglied,\\ & keine Gäste}{}{rohieb}

\title{7.\, Mitgliederversammlung}

\begin{document}
\maketitle

%%%%%%%%%%%%
%% TOP 0  %%
%%%%%%%%%%%%
\section{Protokoll-Overhead}
\begin{description}
  \item[Eröffnung der Versammlung] durch den Vorstandsvorsitzenden um 14:17
  \item[Wahl der Versammlungsleitung:] larsan, einstimmig durch Handzeichen
    bei 1 Enthaltung
  \item[Wahl der Protokollführung:] rohieb, einstimmig durch Handzeichen
  \item[Quoren:] zum Tag der Mitgliederversammlung hat der Verein insgesamt 78
    Mitglieder, davon 66 ordentliche Mitglieder.
    \begin{itemize}
      \item 15{,}1 Mitglieder = 23\% der ordentlichen Mitglieder für
        Beschlussfähigkeit
      \item 13{,}5 Mitglieder = 50\% der anwesenden, stimmberechtigten Mitglieder
        für Annahme eines Antrags
    \end{itemize}
  \item[Beschlussfähigkeit:] 27 von geforderten 15{,}1 (23\%) stimmberechtigten
    Mitglieder anwesend, die Versammlung ist damit beschlussfähig.
  \item[Notation für Abstimmungen:] (Pro-Stimmen/Contra-Stimmen/Enthaltungen)
\end{description}

%%%%%%%%%%%%%
%% TOP 1   %%
%%%%%%%%%%%%%
\section{Berichte und Entlastung des Vorstands}

%%%%%%%%%%%%%
%% TOP 1.1 %%
%%%%%%%%%%%%%
\subsection{Finanzbericht}

chrissi\^{} gibt als Schatzmeister einen Überblick über die Finanzen im
vergangenen Jahr. Der vollständige Bericht ist als Präsentation auf der Homepage
zu finden\footnote{\url{https://stratum0.org/wiki/Datei:Finanzbericht2016.pdf}}
und wird hier auszugsweise mit den mündlichen Anmerkungen wiedergegeben.

\paragraph{Überblick über die Finanzen im Jahr 2016}
Die Zahlen beziehen sich auf den Zeitraum 1.\,Dezember 2015 bis 30.\,November
2016:

\newpage

\begin{longtable}{lr>{\textcolor{red}\bgroup}r<{\egroup}}
  \textbf{Bereich} & \textbf{Einnahmen [€]} & \textbf{Ausgaben [€]} \\
  \midrule
  \endfirsthead
  \multicolumn{3}{c}{\emph{(Fortsetzung von vorheriger Seite)}} \\
  \\
  \textbf{Bereich} & \textbf{Einnahmen [€]} & \textbf{Ausgaben [€]} \\
  \midrule
  \endhead
  \\
  \multicolumn{3}{c}{\emph{(Fortsetzung auf nächster Seite)}} \\
  \endfoot
  \endlastfoot
  Ideeller Bereich: Allgemein       & 12{.}106{,}61 &  -2{.}220{,}52 \\
  \quad davon Mitgliedsbeiträge     & 11{.}314{,}98 &                \\
  \quad davon Spenden               &      445{,}00 &                \\
  \quad davon allgemeine Ausgaben   &               &      -165{,}43 \\
  \quad davon Kontoführungsgebühren &               &      -106{,}03 \\
  \quad davon Vereinsserver         &               &      -678{,}47 \\
  \quad davon Bekleidung            &               &  -1{.}270{,}59 \\
  \midrule
  Ideeller Bereich: Projekte        &      562{,}04 &  -1{.}307{,}50 \\
  \quad davon Bastelmaterial        &               &       -89{,}60 \\
  \quad davon Stickmaschine         &       41{,}07 &      -170{,}95 \\
  \quad davon Schneidplotter        &       45{,}21 &      -133{,}20 \\
  \quad davon 3D-Drucker            &       75{,}76 &      -240{,}31 \\
  \quad davon Freifunk              &      300{,}00 &      -340{,}45 \\
  \quad davon CoderDojo             &      100{,}00 &      -332{,}99 \\
  \midrule
  Ideeller Bereich: Space           &  2{.}119{,}26 & -12{.}712{,}76 \\
  \quad davon Rundfunkgebühr        &               &       -69{,}96 \\
  \quad davon Miete und Nebenkosten &  2{.}119{,}26 & -12{.}289{,}40 \\
  \quad davon Verbrauchsmaterial    &               &      -121{,}60 \\
  \quad davon Einrichtung           &               &      -231{,}80 \\
  \midrule
  Vermögensverwaltung               &        0{,}00 &         0{,}00 \\
  \midrule
  Zweckbetriebe                     &        0{,}00 &         0{,}00 \\
  \midrule
  Wirtschaftlicher Geschäftsbetrieb &  5{.}160{,}48 &  -4{.}700{,}05 \\
  \quad (Matekasse) && \\
  \midrule
  Mankobuchungen                    &        3{,}40 &        -1{,}22 \\
  \midrule\midrule
  Gesamt:                           & 19{.}951{,}79 & -20{.}942{,}05 \\

  \textbf{Gewinn/Verlust Gesamt:}   &               &      -990{,}26 \\
\end{longtable}

\paragraph{Gewinn- und Verlustrechnung} Von den Kassenprüfern wurde wiederholt
angemerkt, dass zweckgebundene Spenden (z.\,B. für Stickmaschine,
Schneidplotter, Freifunk) zeitnah ausgegeben oder zurückerstattet werden
müssen. Das wurde dieses Jahr besser als zuvor verfolgt.

Die Nebenkostenabrechnungen für 2014 sind hier noch nicht berücksichtigt, weil
sie noch nicht von der Hausverwaltung eingefordert wurden. Es handelt sich
hierbei um eine Nachzahlung in Höhe von etwa 1{.}300\,€.

Zum wirtschaftlichen Geschäftsbetrieb wird die Frage gestellt, ob sich die
Zahlen im ähnlichen Bereich wie letztes Jahr bewegen. Dies ist laut chrissi\^{}
der Fall, unterliegt aber natürlichen Schwankungen durch die ungleichmäßige
Frequenz der Einkäufe. Wie auch auf Folie 24 erkennbar ist, liegen die Einnahmen
im Getränkeverkauf weiterhin über den Ausgaben. Einen großen Gewinn lieferte
hier auch die 5-Jahres-Feier im Juni. Die neu eingeführte Beschränkung auf
höchstens 10 unbezahlte Striche auf der Getränke-Strichliste hat sich zudem
positiv ausgewirkt.

In der Gewinn- und Verlustrechnung ist der übliche Geldfluss zu erkennen: wir
finanzieren uns hauptsächlich durch Mitgliedsbeiträge und finanzieren dadurch
hauptsächlich unsere Räumlichkeit.

\paragraph{Saldo im Verlauf}
Folie 14 des Finanzberichts zeigt den Kontosaldo über das Jahr. Hierbei sind
regelmäßige Einbrüche durch die Überweisung der Miete erkennbar, sowie der
kontinuierliche Anstieg durch Mitgliedsbeiträge (an dieser Stelle der
obligatorische Hinweis, dass der Mitgliedsbeitrag am Ersten eines Monats fällig
wird und die Stimmberechtigung auf der Mitgliederversammlung an die Zahlung des
Beitrags gekoppelt ist). Durch eine Änderung der Kontoverbindung der
Schimmel-Hof GmbH, die nicht in unserem Dauerauftrag aktualisiert wurde,
entstanden allerdings rückläufige Überweisungen, die durch eine größere
Überweisung im Herbst abgedeckt wurden. Außerdem ist die Bestellung der T-Shirts
im Februar erkennbar (1{.}270\,€), die durch Vorauszahlungen von Mitgliedern im
vorigen Geschäftsjahr möglich wurde.

Auf Folie 15 wird das Guthaben über den gesamten Lebenszeit des Vereins
dargestellt. Im Jahr 2015 ist als großer Ausschlag das Sponsoring durch VW
erkennbar. Insgesamt bewertet chrissi\^{} die Finanzen als gesund für einen
Verein von unserer Größe, der eine Räumlichkeit unterhält.

\paragraph{Größere Einnahmen und Ausgaben}
Folie 16 zeigt den Verlauf der Mitgliedsbeiträge und Spenden. Der
Monatsdurchschnitt der Mitgliedsbeiträge fällt mit 942\,€ dieses Jahr etwas
geringer aus als letztes Jahr (1{.}002\,€), ebenso das Spendenaufkommen mit
durchschnittlich 37\,€/Monat (letztes Jahr: etwa 300\,€/Monat, inkl. Spenden zum
EasterHegg 2015). Noch nicht berücksichtigt sind dabei aber die noch vor der
Versammlung in bar eingegangen Mitgliedsbeiträge.

Auf Folie 17 wird der Spendenverlauf genauer aufgeschlüsselt. Der Graph ist sehr
unkontinuierlich, da die einzelnen Spendenboxen nur nach Bedarf geleert werden.

Die Gegenüberstellung von monatlichen Einnahmen und regelmäßigen Verpflichtungen
(Folie~20) zeigt dieses Jahr ein geringes Defizit:

\begin{center}
\begin{multicols}{2}
\begin{tabular}{lrr}
  \textbf{Einnahmen} & \multicolumn{2}{c}{\textbf{€/Monat}} \\
  \midrule
  Mitgliedsbeiträge & \diameter &  943 \\
  Spenden           & \diameter &   37 \\
                    &           &      \\
                    &           &      \\
                    &           &      \\
                    &           &      \\
  \midrule
  \textbf{Gesamt} & \textbf{\diameter} & \textbf{980} \\
\end{tabular}
\begin{tabular}{lr}
  \multicolumn{1}{c}{\textbf{Verpflichtungen}} & \textbf{€/Monat} \\
  \midrule
  Miete, Nebenkosten      & 630 \\
  Strom                   & 240 \\
  Internet                &  42 \\
  Server, Domain          &  57 \\
  Haftpflichtversicherung &  12 \\
  Domain stratum0.org     &   1 \\
  \midrule
  \textbf{Gesamt:} & \textbf{982} \\
\end{tabular}
\end{multicols}
\end{center}

Im Vergleich zum Vorjahr ist das Niveau der Mitgliedsbeiträge und Spenden dieses
Jahr niedriger gewesen, was sich im negativen Gesamtsaldo der Gewinn- und
Verlustrechnung widerspiegelt.

\paragraph{Mitgliederentwicklung}
Die Zahl der Mitglieder (Folie 22 und 23) ist allerdings weiterhin ansteigend
und nähert sich der Zahl von 80 Mitgliedern an. Zwischendrin gab es öfters
Ausschlüsse wegen Zahlungsausstand. Hierzu weist der Schatzmeister darauf hin,
dass die Mitglieder, die inzwischen nicht mehr unter die Ermäßigungsregel
fallen, aber noch ermäßigt zahlen, gerne Vollzahler werden dürfen. Zur Frage,
wie viele Mitglieder den ermäßigten Beitrag zahlen, hat der Schatzmeister zu
diesem Zeitpunkt keine Zahlen vorliegen.

\paragraph{Bestände}
Die Kontostände zum Zeitpunkt der Kassenprüfung am 30.\,November (Folie 25)
sehen wie folgt aus:\footnote{siehe auch laufender Finanzreport unter
\url{https://data.stratum0.org/finanz/}}

\begin{center}
\begin{tabular}{lr}
  \textbf{Konto} & \textbf{Saldo [€]} \\
  \midrule
  (B) Barkasse Stratum0                    &     275{,}81 \\
  (D) 3D-Drucker Filamentspenden           &       0{,}00 \\
  (P) Pfand für Physische Schlüssel        &     240{,}00 \\
  (F) Spenden for Plotter-Material         &       3{,}80 \\
  (S) Spenden für Material Stickmaschine   &     179{,}91 \\
  (V) Erstattungskasse Verbrauchsmaterial  &      14{,}02 \\
  (M) Matekasse                            &      33{,}37 \\
  (R) Rückstellungen Giro                  & 3{.}160{,}00 \\
  (G) Business Direct                      & 1{.}776{,}06 \\
  \midrule
  \textbf{Gesamt:}                 & \textbf{5{.}682{,}97} \\
\end{tabular}
\end{center}

Hierbei nochmals die Anmerkung, dass die restlichen Spenden für die Stickmaschine
schnellstens zweckgebunden ausgegeben werden müssen. Die Verbrauchsmaterialkasse
dient weiterhin zur selbstständigen Erstattung beim Kauf von Verbrauchsmaterial
gegen Beleg und wird vom Schatzmeister nach Bedarf auf etwa 40\,€ aufgefüllt.

\paragraph{Rücklagen}
Die Rückstellungen von 3{.}160\,€ ergeben sich wie folgt (Folie 26):

\begin{center}
\begin{tabular}{lr}
  \textbf{Zweck der Rücklage} & \textbf{Betrag [€]} \\
  \midrule
  Erhöhung Mietsicherheit Space 2.0 &     160{,}00 \\
  \qquad (wurde bisher von der Vermieterin nicht eingefordert) & \\
  Puffer Einnahmeausfälle           & 3{.}000{,}00 \\
  \midrule
  \textbf{Gesamt:}          & \textbf{3{.}160{,}00} \\
\end{tabular}
\end{center}

Die Rücklage zur Erhöhung der Mietsicherheit von 160\,€ wurde allerdings auf der
letzten Vorstandssitzung am 2.\,Dezember 2016\footnote{Protokoll:
\url{https://stratum0.org/wiki/Vorstandssitzung_2016-12-02}} wieder aufgelöst,
da sich bisher keine Notwendigkeit ergeben hat und sich der Betrag auch leicht
aus dem Tagesgeschäft entnehmen lässt. Außerdem wurde auf dieser
Vorstandssitzung eine zusätzliche Rücklage von 1{.}\,300\,€ zur Deckung der
Nebenkostenabrechnung 2014 beschlossen, die im Finanzbericht vom 30.\,November
auch noch nicht aufgeführt ist. Die Rücklage von 3{.}000\,€ dient weiterhin zur
Abwicklung der laufenden Geschäfte im Falle von Umsatzeinbrüchen in den
Mitgliedsbeiträgen.

\paragraph{Zusammenfassung und Ausblick}
Insgesamt ist ein leichter Rückgang bei den Mitgliedsbeiträgen und ein
deutlicher Rückgang bei den Spenden sichtbar. Da die Nebenkosten für 2014 mit
1{.}300\,€ angesetzt wurden, ist für 2015 allerdings eine ähnliche Nachzahlung zu
erwarten. Insbesondere die Stromkosten\footnote{Auswertung Strom:
\url{https://data.stratum0.org/energy/energy-360d.png}} bewegen sich oberhalb
von 10\,MWh/Jahr, was sich auch durch die laufende Nutzung der Ventilatoren
im Frickelraum und auf dem Holodeck seit dem Sommer erklären lässt. Die
Nebenkosten bewegen sich allerdings nach der Einschätzung des Schatzmeisters im
durchschnittlichen Bereich für einen Raum unserer Größe. Allerdings sollte man
die gestiegenen Nebenkosten mit etwa 100\,€ mehr im Monat in die laufenden
Verpflichtungen einplanen, und ebenso die Einnahmen um denselben Betrag
steigern (z.\,B. durch Mitgliedsbeiträge, Spenden, oder Partys). Es wird auch
nochmal darauf hingewiesen, dass die vor der Versammlung in bar eingenommenen
Mitgliedsbeiträge noch nicht verbucht sind. Für einige Posten (Stickmaschine,
Freifunk, CoderDojo) sind zudem noch zweckgebundene Spenden auszugeben, die
nicht als freies Kapital verwendet werden können.

Es wird nach den Erlösen aus dem EasterHegg 2015 gefragt. Diese liegen weiterhin
in der SMFW UG (haftungsbeschränkt) und sind daher nicht an gemeinnützige
Verwendung gebunden. Der Plan ist, diese Erlöse als flexiblen Zuschuss nach
Bedarf zu verwenden, u.\,a. auch für die Durchführung des Hacken Open Air
(siehe dazu \ref{sec:hackenopenair}).

\paragraph{Sorgfalt bei „Selbstbedienungskassen“}
Auf Bitten der Rechnungsprüfer sagt chrissi\^{} noch etwas zu den
Selbstbedienungskassen (Matekasse und Verbrauchsmaterialkasse). Diese dienen
dazu, Einkäufe (Getränke, Verbrauchsmaterial) zu erstatten, ohne jedes Mal einen
Vorstandsbeschluss zu erfordern. Es wird hierzu ein Kassenbuch pro Kasse geführt
und entsprechende Einkäufe dürfen gegen Beleg und Eintrag im Kassenbuch selbst
aus der Kasse erstattet werden. Wünschenswert ist hierbei ein Beleg
(Kassenzettel, Rechnung) pro Eintrag im Kassenbuch, um den Rechenaufwand und
damit die Nachvollziehbarkeit für den Schatzmeister und die Rechnungsprüfer
gering zu halten. Im Idealfall befinden sich auf einem Beleg nur Posten des
Vereins. In jedem Fall sollten (unleserliche) Nebenrechnungen als Notiz auf dem
Beleg selbst vermieden werden! Falls im Einzelfall private und vereinsbezogene
Posten zusammen auf einer Rechnung auftauchen, kann von der einkaufenden Entität
auch eine \emph{Ersatzrechnung} gestellt werden, die eine Auflistung der
kompletten Posten des Vereins und deren Summe enthält, und der die
\emph{Originalrechnung(en) als Anhang} beigefügt wird.

%%%%%%%%%%%%%
%% TOP 1.2 %%
%%%%%%%%%%%%%
\subsection{Rechenschaftsbericht}
hanhaiwen trägt aus Sicht des Vorstands die wichtigsten Themen des Jahres vor.

\paragraph{Gründerquartier}
Von der Stadt Braunschweig wurden wir in das neu aufgestellte „Gründerquartier“
\footnote{\url{http://www.braunschweig.de/wirtschaft_wissenschaft/existenzgruendung/gruenderquartier.html}}
aufgenommen. Es handelt sich dabei um ein Netzwerk von Anlaufstellen für
Existenzgründer, in dem u.\,a. auch das Protohaus, das Haus der Wissenschaft
und der Technologiepark am Rebenring verzeichnet sind. Bisher entstehen uns
dadurch keine Verpflichtungen, es dient aber unserer Wahrnehmung in der
Öffentlichkeit.

\paragraph{Zweckgebundene Spenden} Aus zweckgebundenen Spenden wurde ein Set
LEGO Mindstorms für 333€ angeschafft, das im CoderDojo als didaktisches
Spielzeug zum Programmierenlernen für Schüler\_innen eingesetzt werden kann.
Außerdem wurden 172\,€ für mehr Werkzeug und Zubehör zur Textilverarbeitung
ausgegeben.

\paragraph{Maker Faire Hannover} Unsere Beteiligung bei der Maker Faire Hannover
wurde dieses Jahr wieder von Community-Mitgliedern organisiert, der Vorstand war
nicht involviert.

\paragraph{5tratum}
Unsere 5-Jahres-Feier fand im Juni statt und wurde vorerst aus Geldern
der SMFW UG (haftungsbeschränkt) finanziert. Die übrig gebliebenen Getränke
(mehrere Kisten Club-Mate und Club-Mate Cola) wurden danach vom Verein
übernommen, was sich (positiv) im Umsatz der Getränkekasse widerspiegelte.

\paragraph{Sponsoring Triology}
Im Jahr 2017 werden wir von der Triology GmbH durch einer Spende von 1{.}000\,€
unterstützt. Mit der Firma bestand im Rahmen des CoderDojo schon längere
Zusammenarbeit in Bezug auf Räumlichkeiten, und von ihrer Seite wurde auch
gewünscht, diese Zusammenarbeit in Zukunft fortzuführen und womöglich die
Zielgruppe der CoderDojos z.\,B. auch auf ältere Schüler und Studenten
auszuweiten. Dies ist aber nicht als harte Verpflichtung für die finanzielle
Unterstützung zu verstehen. Von seiten einiger Mitglieder wird der Wunsch
geäußert, dass das CoderDojo nicht als „Recruiting-Veranstaltung“ für neue
Triology-Mitarbeiter genutzt wird. Hierbei wird aber von einem anwesenden
Mitglied, das den Kontakt zu Triology vermittelt hatte, Entwarnung gegeben;
Triology sieht sich durchaus als Unterstützer und nicht als Nutznießer von
offenen Gruppierungen (z.\,B. auch der Braunschweiger Java User Group). In jedem
Fall müssen die involvierten CoderDojo-Mentoren entscheiden, wie sie die
Zusammenarbeit mit Triology fortführen wollen. Eine Ausweitung auf ältere
Zielgruppen (eventuell unter einem anderen Namen) würde sich aber vermutlich
auch in Bezug auf Mitgliederwerbung für den Verein positiv auswirken.

\paragraph{Rücktrittsrecht bei Mitgliedsanträgen}
In einem Fall hat der Vorstand dieses Jahr aus Kulanz einem Neumitglied ein
14-tägiges Rücktrittsrecht vom Mitgliedsantrag eingeräumt. Unabhängig davon ist
es natürlich weiterhin jederzeit möglich, Anträge auf Ermäßigung des
Mitgliedsbeitrags an den Vorstand zu stellen.

\paragraph{Nebenkostenabrechnung} Ein spannender Punkt war dieses Jahr die noch
ausstehenden Nebenkostenabrechnungen für die Jahre
2012--2014.\footnote{Chronologischer Verlauf:
\url{https://pad.stratum0.org/p/r.5440a06fc18382eb422ab18e4bc8b2bd}} Für das
Jahr 2013 hatte die Verwaltung schon ein Guthaben von 568\,€ festgestellt, das
aber mit einer eventuellen Nachzahlung von 2012 verrechnet werden sollte. Nach
Fristsetzung unsererseits wurde im September 2016 eine Nachzahlung von 0\,€ für
die Nebenkostenabrechnung 2012 ausgehandelt, das Guthaben von 2013 wurde
daraufhin ausgezahlt. Einer weiteren Fristsetzung unsererseits für die
Nebenkostenabrechnung 2014 wurde von der Verwaltung widersprochen, gleichzeitig
wurden unsere Fragen beantwortet und eine offene Nachzahlung von 1{.}302\,€
festgestellt. Für diesen Betrag wurde nun vom Verein eine Rücklage gebildet, bis
er von der Verwaltung eingefordert wird. Die Nebenkostenabrechnung für das Jahr
2015 ist noch nicht bei uns eingetroffen, hat aber auch noch bis Ende des
laufenden Jahres Zeit.

%%%%%%%%%%%%%
%% TOP 1.2 %%
%%%%%%%%%%%%%
\subsection{Bericht der Rechnungsprüfer}
Angela und shoragan haben die Kasse geprüft und einen fehlenden Beleg
festgestellt, der nachträglich durch einen Ersatzbeleg ersetzt wurde.
Ansonsten wiederholt shoragan seinen Bericht vom letzten Jahr: es gab sonst
keine unerklärlichen Fehler. Er verweist nochmal auf den idealen Umgang mit
Rechnungen bei den Selbstbedienungskassen, die chrissi\^{} schon vorher
angesprochen hatte. Insgesamt war die Buchführung besser als letztes Jahr, auch
zweckgebundene Spenden wurden zielgerichteter ausgegeben. Für den zukünftigen
Schatzmeister gibt es ein paar Kleinigkeiten, die repariert werden sollten, wie
die Zuordnung von einigen Buchungen zu Buchungskategorien. Es gab zudem eine
Menge Hinweise für die automatisierten Regressionstests der Finanzbuchhaltung.
Ein ausführliches Protokoll der Rechnungsprüfung befindet sich im Git-Repository
der Buchhaltung.

Die Rechnungsprüfer sehen somit keinen Grund, eine Entlastung des Schatzmeisters
zu verweigern.

%%%%%%%%%%%%%
%% TOP 1.3 %%
%%%%%%%%%%%%%
\subsection{Entlastung des Vorstandes}
Es wird gefragt, ob eine Einzelentlastung der Vorstandsmitglieder gewünscht
wird. Niemand der Anwesenden wünscht dies. Der Vorstand wird also als ganzes
entlastet.

Valodim und Emantor beantragen die Entlastung des Vorstandes. Es wird per
Handzeichen abgestimmt. 

\begin{resolution}{MV 2016-01}{\vote{\adopted}{21}{0}{6}}{Entlastung des
  Vorstandes}{}
  Alle Vorstandsmitglieder enthalten sich.
\end{resolution}

Der Vorstand ist damit entlastet.

%%%%%%%%%%%%%
%% TOP 1.4 %%
%%%%%%%%%%%%%
\subsection{Jahresbericht}
larsan gibt einen (kulinarischen) Jahresrückblick mit vielen Bildern, die
Präsentation dazu ist auf der Homepage zu
finden.\footnote{\url{https://stratum0.org/wiki/Datei:Jahresbericht2016.pdf}}

Wichtige Punkte dieses Jahr im Wortprotokoll:

Neue Fenster im Frickelraum, Nebenkosten, Wechsel der Hausverwaltung: Die
Gottschalk Identität: NOWO Immobilien, Die Gottschalk Verschwörung: BAHR, Das
Gottschalk Ultimatum: WAG Salzgitter. Brief der Eigentümerin als
„Entschuldigung“, Nebenkostenabrechnung 2015…? Aprilscherz auf der Homepage mit
Stock Photos, Gründerquartier Braunschweig, es wurde Bier gebraut, es wurde
Wurst gemacht, es wurde 5 Jahre Stratum~0 gefeiert (CYBER, CYBER, CYBIER), es
wurde gegessen, es wurde gegessen, es wurde gecampt, es wurde gegessen, es wurde
mehr gegessen, es wurde gegessen. Filmcrew: Imagefilm zum Gründerquartier,
Präsenz auf der MakerFaire, wir wurden dort gefragt „Was verkauft ihr denn
eigentlich?“ – „Äh, Freizeit?“ Neuer Vereinsserver mit vielen virtuellen
Maschinen, sogar ein paar sinnvollen,
Transparenzportal\footnote{\url{https://data.stratum0.org}} könnte noch
gedeihen, Stromverbrauch ist gediehen, es wurde gegessen, es wurde vorgetragen,
es gab Aufzeichnungen, aber es mangelt an verfügbarer Hardware und Zeit für die
Nachbearbeitung. Clap for comawill.

\meetingbreak{von 12 Minuten}

%%%%%%%%%%%%%
%% TOP 2   %%
%%%%%%%%%%%%%
\section{Beschluss einer Geschäftsordnung}

rohieb schlägt vor, die Wahlen in der Geschäftsordnung der Mitgliederversammlung
zu regeln. Das Verfahren der letzten Jahre hat sich bewährt und soll somit
dokumentiert werden. Ein Entwurf wird diskutiert, kritisiert, angepasst und
abgestimmt.\footnote{Abgestimmte Version des Entwurfs:
\url{https://stratum0.org/mediawiki/index.php?oldid=14462}, Rechtschreibfehler
wurden nachträglich im Zuge dieses Protokolls korrigiert.}

\begin{resolution}{MV 2016-02}{\vote{\adopted}{27}{0}{0}}{Beschluss einer
  Geschäftsordnung}{}
  Beschlusstext:

  \begin{quote}
    \itshape
    \textbf{0. Wahlen}
    \begin{enumerate}
      \item[1.] Als Wahlmodus wird die Zustimmungswahl festgelegt. (Jede
        stimmberechtigte Entität kann für jede kandidierende Entität eine Stimme
        abgeben, Stimmenkumulation ist nicht möglich.)
      \item[2.] Es wird in geheimer Wahl auf Stimmzetteln gewählt. Dabei dürfen
        nur von der Versammlungsleitung genehmigte Stimmzettel genutzt werden.
      \item[3.] Die Kandidaten mit den meisten Stimmen sind gewählt, sofern sie
        mindestens die Stimmen von 50\% der Stimmberechtigten erhalten haben,
        solange genug Posten für das entsprechende Amt zu besetzen sind.
      \item[4.] Falls mehrere Posten für ein Amt zu vergeben sind (z.B.
        Beisitzer, Rechnungsprüfer), findet die Besetzung absteigend nach
        Stimmenanzahl statt, bis alle Posten des entsprechenden Amtes besetzt
        sind.
      \item[5.] Falls sich durch Stimmengleichheit keine eindeutige Besetzung
        ergibt, findet eine Stichwahl zwischen den entsprechenden Kandidaten mit
        der gleichen Stimmenanzahl statt. Eine Nachwahl findet jeweils für die
        Posten Schatzmeister, stellv. Vorsitzender, oder Vorsitzender statt,
        falls das entsprechende Amt nicht besetzt wurde.
      \item[6.] Die Auswertung der Wahl erfolgt in der Reihenfolge
        Vorstandsvorsitzender, stellv. Vorsitzender, Schatzmeister, Beisitzer,
        Rechnungsprüfer. Kandidaten, die schon für ein Amt gewählt worden sind
        und es angenommen haben, werden bei der Auswertung der nachfolgenden
        Ämter nicht mehr berücksichtigt.
      \item[7.] Gewählte Kandidaten können von der Wahl zurücktreten. Die
        Annahme der Wahl oder der Rücktritt von der Wahl ist auch fernmündlich
        möglich.
    \end{enumerate}
  \end{quote}
\end{resolution}

Nachdem nun der Wahlmodus geregelt ist, kann gewählt werden.

%%%%%%%%%%%%%
%% TOP 3   %%
%%%%%%%%%%%%%
\section{Wahlen}

Als Wahlleitung wird ktrask durch Handzeichen ohne Gegenstimmen oder
Enthaltungen gewählt. Die Versammlungsleitung wird an die Wahlleitung übergeben.
Die Wahlleitung erklärt den oben beschlossenen Wahlmodus nochmal in eigenen
Worten.

Jedes stimmberechtigte Mitglied hat vor Beginn der Veranstaltung einen
Stimmzettel erhalten. Die Kandidaten, die sich vor der Veranstaltung schon zu
einer Kandidatur entschieden hatten, sind schon auf den Stimmzetteln vorhanden.
Es wird die Frage gestellt, ob es noch weitere Kandidaten gibt. whisp möchte
spontan für ein Beisitzeramt kandidieren und füllt damit Platz Nr. 8 auf dem
Stimmzettel. Niemand weiteres meldet sich, damit sieht die Kandidatenliste wie
folgt aus:

\begin{center}
\begin{multicols}{2}
\begin{tabular}{rl}
  \multicolumn{2}{c}{\bfseries Vorstandsvorsitzender} \\
  \midrule
  1 & larsan \\
  2 & Kasalehlia \\
  \\
  \multicolumn{2}{c}{\bfseries stellv. Vorstandsvorsitzender} \\
  \midrule
  1 & larsan \\
  2 & rohieb \\
  \\
  \multicolumn{2}{c}{\bfseries Rechnungsprüfer} \\
  \midrule
  1 & sonnenschein \\
  2 & shoragan \\
  \\
\end{tabular}
\begin{tabular}{rl}
  \multicolumn{2}{c}{\bfseries Schatzmeister} \\
  \midrule
  1 & Emantor \\
  \\
  \multicolumn{2}{c}{\bfseries Beisitzer} \\
  \midrule
  1 & chrissi\^{} \\
  2 & larsan \\
  3 & Kasalehlia \\
  4 & rohieb \\
  5 & reneger \\
  6 & hanhaiwen \\
  7 & Chaosgrille \\
  8 & whisp \\
\end{tabular}
\end{multicols}
\end{center}

Die einzelnen Kandidaten stellen sich kurz vor.

Die Wahlurne wird ausgeleert, herumgezeigt und versiegelt. Der erste Wahlgang
wird um 16:07 eröffnet. Nachdem niemand mehr einen Stimmzettel einwerfen will,
wird der Wahlgang um 16:11 geschlossen. Die Wahlleitung sucht sich drei
anwesende Mitglieder, die nicht zur Wahl stehen, als Wahlhelfer. Deren
Integrität wird von der Versammlung nicht angezweifelt. Die Wahlleitung zieht
sich zur Auszählung zurück.

\meetingbreak{zur Auszählung bis 16:35}

Nach der Pause gibt die Wahlleitung das Ergebnis bekannt. Es gab 27 gültige
Stimmzettel. Das Quorum von 50\% Zustimmung liegt bei mindestens 14 Stimmen.

\begin{center}
\begin{multicols}{2}
\begin{tabular}{r l l@{ Stimmen (}c@{\%)}}
  \multicolumn{4}{c}{\bfseries Vorstandsvorsitzender} \\
  \midrule
  1 & larsan       & 20 & 74.1 \\
  2 & Kasalehlia   & 20 & 74.1 \\
  \\
  \multicolumn{4}{c}{\bfseries stellv. Vorstandsvorsitzender} \\
  \midrule
  1 & larsan       & 20 & 74.1 \\
  2 & rohieb       & 25 & 92.5 \\
  \\
  \multicolumn{4}{c}{\bfseries Rechnungsprüfer} \\
  \midrule
  1 & sonnenschein & 24 & 88.8 \\
  2 & shoragan     & 22 & 81.5 \\
\end{tabular}
\begin{tabular}{r l l@{ Stimmen (}c@{\%)}}
  \multicolumn{4}{c}{\bfseries Schatzmeister} \\
  \midrule
  1 & Emantor      & 27 & 100 \\
  \\
  \multicolumn{4}{c}{\bfseries Beisitzer} \\
  \midrule
  1 & chrissi\^{}  & 25 & 92.5 \\
  2 & larsan       & 21 & 77.7 \\
  3 & Kasalehlia   & 22 & 81.5 \\
  4 & rohieb       & 19 & 70.4 \\
  5 & reneger      & 19 & 70.4 \\
  6 & hanhaiwen    & 18 & 66.7 \\
  7 & Chaosgrille  & 12 & 44.4 \\
  8 & whisp        & 14 & 51.8 \\
\end{tabular}
\end{multicols}
\end{center}

\begin{description}
  \item[Vorstandsvorsitzender:] Es gibt bei der Wahl des Vorstandsvorsitzenden
    einen Gleichstand zwischen larsan und Kasalehlia. larsan zieht seine
    Kandidatur an dieser Stelle zurück. Kasalehlia nimmt die Wahl an.
  \item[Stellv. Vorstandsvorsitzender:] rohieb nimmt die Wahl an.
  \item[Schatzmeister:] Emantor nimmt die Wahl an.
  \item[Beisitzer:] chrissi\^{} nimmt die Wahl an. Kasalehlia ist schon als
    Vorstandsvorsitzender gewählt und scheidet damit aus. larsan nimmt die Wahl
    an.  rohieb ist schon als stellv. Vorstandsvorsitzender gewählt. reneger
    nimmt die Wahl an.
  \item[Rechnungsprüfer:] sonnenschein und shoragan nehmen die Wahl an.
\end{description}

\begin{resolution}{MV 2016-03}{}{Wahlen}{}
  Der neue Vorstand besteht somit aus:

  \begin{electionblock}
    \elected{Vorstandsvorsitzender}{Kasalehlia (Hilko Boekhoff)}{74.1\%}
    \elected{stellv. Vorstandsvorsitzender}{rohieb (Roland Hieber)}{92.5\%}
    \elected{Schatzmeister}{Emantor (Rouven Czerwinski)}{100\%}
    \elected{Beisitzer}{chrissi\^{} (Chris Fiege)}{92.5\%}
    \elected{Beisitzer}{larsan (Lars Andresen)}{77.7\%}
    \elected{Beisitzer}{reneger (René Stegmaier)}{70.4\%}
  \end{electionblock}

  Als Rechnungsprüfer, die nicht zum Vorstand zählen, wurden gewählt:

  \begin{electionblock}
    \elected{Rechnungsprüfer}{sonnenschein (Angela Schmitt)}{88.8\%}
    \elected{Rechnungsprüfer}{shoragan (Jan Lübbe)}{81.5\%}
  \end{electionblock}
\end{resolution}

Die Wahlen sind damit beendet. larsan dankt der Wahlleitung und den Wahlhelfern,
und übernimmt wieder die Versammlungsleitung.

%%%%%%%%%%%%%
%% TOP 4   %%
%%%%%%%%%%%%%
\clearpage
\section{Änderungen der Beitragsordnung}

Der Antrag "`Monatliche Beiträge sollen glatte Centbeträge ergeben"' wurde von
rohieb
eingebracht.\footnote{\url{https://github.com/stratum0/stratum0-dokumente/pull/21}}

\paragraph{Begründung:} Fördermitglieder zahlen einen Jahresbeitrag, ordentliche
Mitglieder zahlen einen monatlichen Beitrag. Die Sollbuchung aller
Mitgliedsbeiträge erfolgt aber im Schatz\-meister-Workflow monatlich, da dies den
kleinsten gemeinsamen Nenner für die Fälligkeit der Beiträge darstellt. Für den
Workflow ist es nicht praktikabel, Fördermitglieder gesondert zu behandeln;
Jahresbeiträge werden hierbei auf den entsprechenden monatlichen Anteil
umgerechnet. Die Beschränkung der Jahresbeiträge auf glatt durch 12 teilbare
Centbeträge soll nun vermeiden, dass bei dieser Umrechnung Centbeträge mit
Nachkommastellen entstehen, die zu Rundungsfehlern führen würden, nicht
überwiesen werden könnten und zusätzliche Ausgleichsbuchungen erfordern würden.

Eine ähnliche Formulierung soll auch für den ermäßigten Beitrag erfolgen, nur um
auf der sicheren Seite zu sein.

\question{Ist es sinnvoll, ein Symptom zu behandeln, statt solche
Beiträge lieber jahresbezogen als einmalige Spenden zu behandeln?}
\answer[chrissi\^{}]{Nein, denn es handelt sich hier um Mitgliedsbeiträge, nicht
um Spenden. Spenden werden nicht durch Sollbuchungen erfasst, da hier kein
Gegenkonto vorliegt. Es geht hier auch nur um die Verbuchung auf den
entsprechenden Mitgliedskonten, auf welchem Wege die Beiträge bezahlt werden,
ist hier irrelevant. Im Moment erfordert der Schatzmeister-Workflow monatliche
Ausgleichsbuchungen, um den korrekten Betrag am Ende des Jahres darstellen zu
können. Bisher mussten die Beiträge der einzelnen Fördermitgliedschaften so
verhandelt werden, dass sie im Workflow abbildbar sind, die neue Regelung würde
Arbeitserleichterung für den Vorstand bringen.}

\question{Um wie viele Fördermitgliedschaften handelt es sich im Moment?}
\answer[chrissi\^{}]{Aktuell gibt es 12 Fördermitglieder, Tendenz steigend.}

Die vorgestellte Änderung wird auf dem Beamer gezeigt, vorgelesen und zur
Abstimmung gestellt:

\begin{resolution}{MV 2016-04}{\vote{\adopted}{21}{1}{5}}{Änderung der
  Beitragsordnung: monatliche Beiträge sollen glatte Cent\-beträge ergeben}{}
  §0 (Beitragssätze), Abs.\,1 der Beitragsordnung wird wie folgt gefasst:
  \begin{quote}
    Der reguläre Mitgliedsbeitrag für ordentliche Mitglieder beträgt 20€ pro
    Monat. Fördermitglieder zahlen einen frei wählbaren Beitrag von mindestens
    30€ pro Jahr. \emph{Der Jahresbeitrag muss ein Vielfaches von 12 Cent
    betragen.}
  \end{quote}

  §0 (Beitragssätze), Abs.\,3 wird wie folgt gefasst:
  \begin{quote}
    Sollte ein ordentliches Mitglied aus finanziellen Gründen den
    Mitgliedsbeitrag nicht aufbringen können, kann dieses beim Vorstand einen
    Antrag auf Ermäßigung oder Befreiung stellen. Diese gilt für maximal ein
    Jahr und kann dann durch einen neuen Antrag erneuert werden. \emph{Der
    ermäßigte monatliche Beitrag muss ein glatter Centbetrag sein.}
  \end{quote}

  (Änderungen kursiv)
\end{resolution}

%%%%%%%%%%%%%
%% TOP 5   %%
%%%%%%%%%%%%%
\section{Sonstiges}

%%%%%%%%%%%%%
%% TOP 5.1 %%
%%%%%%%%%%%%%
\subsection{Hacken Open Air}\label{sec:hackenopenair}
Kasalehlia stellt das Hacken Open Air vor. Es ist geplant, dieses Jahr vom
Verein aus ein eigenes Sommercamp auszurichten, ähnlich dem Chaos Communication
Camp. Dazu gibt es Kontakte zum Jugendzentrum Peine, das auch die Örtlichkeit
stellen wird. Die Orga besteht im Moment aus Kasalehlia, Daniel vom
UJZ Peine, und reneger; es ist geplant, die angeschafften Geräte der
SMFW UG zu nutzen. Wer sich beteiligen will, ist gern gesehen, die Orga-Phase
soll demnächst starten und es wird noch eine gesonderte Mail dazu geben.

Es wird die Frage gestellt, ob der Vorstand beschlossen hat, das Camp
durchzuführen. Dies ist nicht der Fall, Überlegungen in dieser Hinsicht wurden
aber schon auf der letzten Mitgliederversammlung angekündigt.

%%%%%%%%%%%%%
%% TOP 5.2 %%
%%%%%%%%%%%%%
\subsection{Termin nächste Mitgliederversammlung}
Die jährliche Mitgliederversammlung fand bisher immer im Dezember statt. Der
Termin ist durch die Satzung vorgegeben, die eine Mitgliederversammlung einmal
im Jahr vorschreibt, und die Amtszeiten der Vorstandsmitglieder auf ein Jahr
begrenzt. Da das Geschäftsjahr aber mit dem Kalenderjahr übereinstimmt, ist bei
einer Mitgliederversammlung im Dezember immer ein unvollständiges Geschäftsjahr
berücksichtigt worden, sodass de facto ein vorläufiger Jahresabschluss im
Dezember zur Mitgliederversammlung und ein weiterer am Ende des Jahres (mit der
neuen Amtsperiode) stattfand. Außerdem ist die Zeit vor Weihnachten und vor dem
jährlichen Chaos Communication Congress erfahrungsgemäß stressig. Eine
Mitgliederversammlung im Januar hätte in dieser Hinsicht Vorteile.

Hinsichtlich der Begrenzung der Vorstands-Amtszeit auf (höchstens) ein Jahr wäre
eine zwischengeschobene Vorstandswahl in der Mitte des Jahres denkbar.

Es wird kurz per Handzeichen ein Meinungsbild eingeholt. Die anwesenden
Mitglieder sind überwiegend dafür, den Termin in den Januar zu verschieben.

\begin{resolution}{MV 2016-05}{\consensus{\adopted}}{Termin der jährlichen
  Mitgliederversammlung in den Januar verschieben}{}
\end{resolution}

Es wird Sonntag, der 14.\, Januar 2018 als Termin in den Raum gestellt. Der
Vorstand wird rechtzeitig einen Termin bestimmen und mit genügend Vorlauf
ankündigen.

\meetingend{17:08}
\end{document}

% vim: set et ts=2 sw=2 sts=2 :
